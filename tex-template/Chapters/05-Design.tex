\chapter[Diseño de Solución]{Diseño de Solución}
\label{cp:design}

\parindent0pt

- Los requerimientos func y no func son los fundamentos a partir de los cuales se diseña la solución.
- El diseño de la solución en el modelo en v tiene dos fases: diseño de módulos y diseño de componentes.
- Qué contempla cada una de estas etapas de diseño. Diferencias entre ambas, orden.
- El diseño permite una visión global de la solución, asegurando que todos los componentes y módulos trabajen juntos de manera coherente previo a la implementación. 
- La importancia de esta etapa esta en que si te la saltas luego el desarrollo se vuelve más complicado y propenso a errores, gaps, inconsistencias.
- Cada etapa de diseño tiene su correspondiente etapa de pruebas en el otro lado de la v.
- Los req no funcionales se contemplan en el diseño de modulos, aunque no tengan user story.
- Los req funcionales se implementan siguiendo las historias de usuario, y se contemplan en el diseño de módulos y componentes.
- A continuación se detalla en cada seccion cada uno de los diseños.

\section{Diseño de Módulos}
\label{sec:module-design}

- Esta etapa se centra en tal cosa y es la primera, su entrada son los requerimientos, su salida es la arquitectura del sistema, que comprende módulos, interacciones y tecnologías.
- Contar sobre la etapa de pruebas correspondiente del otro lado.
- Criterios para separar módulos y definir las interfaces (basicamente usamos los estandares de la industria para apps webs de modo que en el futuro sea extensible, mantenible e integrable con otras fuentes de datos como iot en los procesos productivos).
- Empezar a contar: Primero armé la arquitectura del sistema, separando en bloques frontend/api/datos. 
- Contar que decidí separar los módulos segun rol y los compartidos aparte por el acoplamiento y separación de responsabilidades.
- Contar para cada módulo la comparación y elección de tecnologías. 
- Contar la comparación de blockchains y mencionar el apendice, justificar la elección de blockchain.
- Explicar también las interfaces entre módulos (API rest en gral), pero si usé alguna librería para estas interfaces explicar cuál.
- Contar sobre las combinación de blockchain y sql, y cómo se relacionan los datos de la blockchain con los datos de la base de datos sql.

\section{Diseño de Componentes}
\label{sec:components-design}

- Esta etapa es la segunda y se centra en tal cosa, su entrada son los módulos y requerimientos y su salida es la arquitectura de componentes, que comprende estructura de pantallas, apis y contratos.

- Contar que elegi arch de módulos MVC en front, por qué y mostrar la división de funcionalidades por pantalla.
- Contar en el front que también se eligió el sistema de diseño y paleta de colores, íconos, disposición del sistema y personalidad. Se decidió que sea una plataforma web, no mobile.
- Contar que elegí clean arch en back, por qué y mostrar la división de funcionalidades por módulo. Mostrar la división de dominio.
- Contar cómo armé la arquitectura de contratos en la blockchain y el modelo de datos en sql para un trackeo eficiente, confiable y no redundante de información. Mostrar los DER y reexplicar en detalle su interacción. Contar qué cubre cada contrato y cada tabla.
- Explicar como el repository+handler unifica ambas cosas


- Transicionar al siguiente cap. Que es la implementación, que es la etapa de desarrollo de la solución, donde se implementan los módulos y componentes diseñados en las etapas anteriores. Que haber hecho este diseño permite una implementación más fluida y organizada, ya que se cuenta con una guía clara de cómo deben interactuar los diferentes componentes y módulos.