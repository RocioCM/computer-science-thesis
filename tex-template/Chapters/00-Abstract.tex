\thispagestyle{plain}
\chapter*{Resumen}

La trazabilidad permite identificar el origen y las etapas de producción y distribución de bienes, facilitando la implementación de prácticas de economía circular, donde los residuos se reciclan o reutilizan en lugar de desecharse. En particular, es deseable poder realizar la trazabilidad del vidrio, dado que es un producto que puede ser reciclado o reinsertado en la cadena de suministro de diferentes formas.

Para proporcionar un nivel superior de transparencia, seguridad y eficiencia, los sistemas de trazabilidad están comenzando a hacer uso de la tecnología blockchain. Esta tecnología permite crear registros inmutables y descentralizados, asegurando la integridad de la información y evitando manipulaciones externas. Además, brinda confianza a los consumidores al garantizar la autenticidad y calidad de los productos, mientras que también permite a las organizaciones que adoptan esta tecnología diferenciarse en el mercado, al demostrar su compromiso con la sostenibilidad y el respeto al medio ambiente.

En este trabajo se desarrolla un prototipo de sistema de trazabilidad de envases de vidrio basado en tecnología blockchain, diseñado para registrar y verificar cada etapa de su ciclo de vida, desde la producción hasta su reintroducción en la cadena de valor, facilitando su valorización. Este desarrollo sigue un proceso de ingeniería de software bajo el modelo en V, el cual articula de manera rigurosa las fases de diseño, implementación y pruebas del sistema. A lo largo del presente trabajo, se profundiza en las etapas de análisis de requisitos, diseño arquitectónico, implementación del prototipo y verificación exhaustiva de sus funcionalidades, con el fin de demostrar la viabilidad y los beneficios de aplicar blockchain para una economía circular de vidrio transparente y sostenible.

\pdfbookmark[1]{Abstract}{abstract}
\chapter*{Abstract}

Traceability enables the identification of the origin and various stages of production and distribution of goods, facilitating the implementation of circular economy practices where waste is recycled or reused instead of discarded. In particular, it is desirable to achieve the traceability of glass, as it is a product that can be recycled or reinserted into the supply chain in different ways.

To provide a superior level of transparency, security, and efficiency, traceability systems are beginning to leverage blockchain technology. This technology allows for the creation of immutable and decentralized records, ensuring data integrity and preventing external manipulation. Furthermore, it fosters consumer confidence by guaranteeing product authenticity and quality, while also enabling organizations that adopt this technology to differentiate themselves in the market by demonstrating their commitment to sustainability and environmental responsibility.

This work develops a prototype blockchain-based glass traceability system, designed to record and verify each stage of its lifecycle, from production to its reintroduction into the value chain, thus facilitating its valorization. This development follows a V-model software engineering process, which articulates the design, implementation, and testing phases of the system. The stages of requirements analysis, architectural design, prototype implementation, and exhaustive testing of its functionalities are detailed throughout the work, with the aim of demonstrating the viability and benefits of applying blockchain for a transparent and sustainable circular glass economy.
