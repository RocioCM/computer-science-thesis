\chapter{Entrevista a Verallia}
\label{cp:verallia-interview}

\parindent0pt

% TODO: redactar y formatear esta sección
% Bajar minuta de la entrevista con la chica de Verallia. Contar cuánto duró y lista de preguntas y respuestas, tipo de entrevista.

Para obtener una comprensión profunda de la situación del reciclaje de vidrio, particularmente en la provincia de Mendoza, se realiza una investigación de campo.
Como punto de partida, se lleva a cabo una revisión en internet de sitios web, artículos de diarios locales y boletines oficiales para documentar programas de reciclaje activos en la región y la presencia de empresas productoras relevantes.
A continuación, se realiza una entrevista semiestructurada con Lucía J., responsable del Área de Medio Ambiente de Verallia, la principal empresa productora de envases de vidrio en la región de Mendoza.
La entrevista se condujo de manera telefónica, utilizando preguntas guía predefinidas, pero con flexibilidad para explorar temas emergentes durante la conversación.
La conversación fue grabada con consentimiento previo y luego transcripta, complementándose con notas tomadas durante la entrevista.

---

Durante la etapa de investigación y recopilación de información para el desarrollo de este trabajo, luego de definir el alcance, se llevó a cabo una entrevista con una representante de Verallia, la principal empresa fabricante y manufacturadora de la fabricación de envases de vidrio en Mendoza, principal proveedora de la industria vitivinicola local.

La entrevista se realizó con Lucía, quien es parte del equipo de sostenibilidad de Verallia Argentina en Mendoza. Previo a la entrevista fue informada sobre el contexto de este trabajo y la finalidad de la entrevista y dio consentimiento para que la información compartida fuera incluida en este trabajo. 

La conversación se centró en conocer los procesos actuales de producción y reciclaje de vidrio en Verallia, las iniciativas sostenibles que la empresa está implementando y las oportunidades para mejorar la trazabilidad y la gestión de residuos en la cadena de suministro de la empresa.

La entrevista fue realizada por teléfono, tuvo una duración aproximada de 45 minutos y fue grabada. La modalidad de la entrevista fue semiestructurada, permitiendo una conversación fluida y la posibilidad de profundizar en temas relevantes a medida que surgían. Previo a la entrevista, se preparó una lista de preguntas que abarcaban los siguientes temas:

-
- 
-

A continuación se presenta un resumen de las preguntas realizadas y las respuestas obtenidas:

% Contarlo de forma fluida, no como lista de preguntas y respuestas.
Preguntas:
- Contame cómo es el proceso que hacen ustedes
- El vidrio reciclado de dónde sale
- Qué hacen cuando les llega el vidrio, viene clasificado o lo clasifican ustedes?
- Cómo se involucran sus clientes
- Hay alguna reglamentación o restricción para poder reciclar?
- Les serviría que el vidrio llegara en cantidad y preclasificado?
- Qué les interesaría de un sistema como este? Qué le serviría a tu empresa? Qué tanto les serviría un sistema de trazabilidad? Creen que sería factible incorporar un sistema de estas características?
- Cuánto vuelve en relación a lo que venden?
- Sabés si alguno de tus clientes tienen algún programa particular o compromiso con lo sustentable.

Respuestas:
Recolectores urbanos recogen vidrio.
No hay cultura de reciclado en Arg.
En el camion va junto y en la planta lo separan.
En otros paises hasta el usuario separa.
No hay una politica gral en Mza, es por cada municipio.
A ellos les vende el vidrio los recolectores en negro. Necesitan que sean en blanco. Tienen 2 empresas que cuando compran el vidrio lo limpian. La limpieza es manual y cara. Tienen su nueva propia planta de limpieza.
No se puede reciclar vidrios de distintos colores juntos.
V reciclan botellas y frascos. 
El vidrio limpio va directo a la linea de producción.

V accion transparente es puertas afuera y Lucia es puertas adentro. Hay 32 campanas de la acción trasparente. El muni retira y ellos compran y reciben.
Campañas de entradas al cine.
3 o 4 eventos al año. 
También fomentan que los empleados reciclen.
Concientizar y fomentar el reciclado.
Las campañas dan un aporte mínimo de vidrio.
No acompañan las políticas. 
Mucho viene de recicladores y pymes. Bodegas a veces traen poco y las campañas traen poco.
Sale más barato la materia prima que comprar y limpiar vidrio para reciclaje.
Intermediarios por los que están en negro.
Tienen que clasificar ellos. A veces descartan mucho por los altos niveles de plomo.
La planta de limpieza clasifica. Separan por color y por no sé qué más. Mati de reciclaje. Separan tapas etiquetas basura. No saben de dónde vienen. Competencia Katorini? 
Hay suficiente oferta
Cuál es la legislación para comprar vidrio? Es interna o internacional? Podría venir de greenly la entrega (centralizamos a los recicladores). Estilo punto verde.

Los clientes suyos querrían sumarse a una campaña? A ustedes les serviría? Vendría ya clasificado. Podrían hacerse convenios.

Cuánto fue el objetivo este año? Menor al del año pasado.


- Les serviría un sistema de trazabilidad?
- La empresa podría participar localmente de un experimento o no se puede?Sí
- Les ponen marcadores impresos a las botellas? Eso puede servir para después reciclar y pagás mejor tu vidrio.
- Qué nivel d trazabilidad tienen los envases? Tienen numero de molde y la marca abajo
- La limpieza siempre es necesaria antes re-usarse el calcín
