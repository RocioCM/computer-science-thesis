\chapter{Resultados de Pruebas Realizadas}
\label{cp:tests-execution-results}

En este trabajo, cuyo desarrollo siguió la metodología en V, se realizaron múltiples instancias de pruebas del prototipo tecnológico: pruebas unitarias, pruebas de integración, pruebas de sistema y pruebas de aceptación. Cada etapa incluyó la definición y ejecución de casos de pruebas, ya sean manuales o automatizadas, con el objetivo de validar que el sistema cumpliera con los requisitos funcionales y no funcionales establecidos en la fase de modelado de requerimientos. En este anexo se detallan los casos de uso de cada etapa de pruebas, los resultados obtenidos y las incidencias detectadas y resueltas durante el proceso.

\section{Pruebas Unitarias}

En la fase de pruebas unitarias, se desarrollaron y ejecutaron pruebas automatizadas para cada componente del sistema. Estas pruebas se centraron en validar la funcionalidad de los contratos inteligentes, la API backend y la interfaz frontend. Cada prueba fue diseñada para cubrir casos positivos y negativos, asegurando que cada unidad de código funcionara correctamente de manera aislada. En la Tabla \ref{tab:unit-tests-blockchain} se presenta un listado de las pruebas unitarias realizadas sobre los contratos inteligentes, en la Tabla \ref{tab:unit-tests-backend} las pruebas ejecutadas sobre la API backend, y en la Tabla \ref{tab:unit-tests-frontend} las pruebas realizadas sobre la interfaz frontend.


\begin{xltabular}{\textwidth}{@{} L{1.5cm} Y L{2.5cm} @{}}
	\caption{Listado de pruebas unitarias realizadas sobre los contratos inteligentes}
	\label{tab:functional-requirements}\\
	\toprule
	Número & Título & Resultado \\
	\midrule
\endfirsthead

\toprule
Número & Título & Resultado \\
\midrule
\endhead

\midrule
\multicolumn{4}{r}{\footnotesize Continúa en la siguiente página}
\\\bottomrule
\endfoot

\bottomrule
\endlastfoot

1 & Vender lote de material reciclado exitosamente & \testSuccess \\


\end{xltabular}



Detallar también la lista de casos de prueba de tests de integración.

Detallar la lista de pruebas manuales realizadas.

Contar la lista de bugs levantados y resueltos o desestimados.
