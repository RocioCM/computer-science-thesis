\chapter{Viaje de Investigación}
\label{cp:europe-trip}

Durante la ejecución de este trabajo de tesis, el proceso de investigación fue complementado por una experiencia de campo que, aunque ajena a la planificación inicial, resultó ser enriquecedora para el trabajo. En noviembre de 2024, se realizó un viaje de investigación a Europa con el objetivo de estudiar de primera mano los sistemas de reciclaje y los modelos de economía circular implementados en diversas ciudades. Esta oportunidad permitió obtener una perspectiva global que, si bien generó un desvío en el cronograma, aportó un conocimiento valioso y una visión práctica del problema que se aborda en este trabajo.

La experiencia de dos semanas de duración incluyó visitas a centros de reciclaje, interacciones con organizaciones dedicadas a la sostenibilidad y el estudio de redes de comercio circular en tres ciudades: Madrid (España), Ámsterdam (Países Bajos) y Berlín (Alemania). Cada una de ellas ofreció un panorama distinto de los esfuerzos por promover la economía circular. La primera ciudad visitada fue Madrid, donde se observó la mecánica de la recolección diferenciada y las categorías de residuos en origen. A pesar de que la sociedad española muestra un creciente interés por la sostenibilidad, se constató que la implementación de sistemas de reciclaje y economía circular aún se encuentra en una etapa de desarrollo inicial, en comparación con las ciudades visitadas posteriormente. Se analizó el rol de Ecoembes \footnote{https://www.ecoembes.com/es}, una organización sin fines de lucro encargada de la gestión de los envases domésticos, cuyo modelo de financiamiento a través de las empresas envasadoras permite costear el proceso de reciclaje. A pesar de su importancia, se hallaron controversias sobre las cifras de reciclaje que reportan, un debate que pone de manifiesto la complejidad de la medición en estos sistemas.

El viaje continuó en Ámsterdam, una ciudad reconocida por su alto nivel de conciencia ambiental y sus esfuerzos por fomentar hábitos de vida sostenibles. Aterrizando en el aeropuerto de la ciudad, un gran parque de energía eólica fue visible desde las alturas, una muestra inicial del compromiso de la ciudad con las energías limpias. Se observó el uso extendido de bicicletas y un sistema de transporte público mayoritariamente eléctrico que contribuyen a reducir la huella de carbono de sus habitantes. En lo que respecta al reciclaje de envases, Ámsterdam implementa un sistema de depósito y retorno (DRS), en el que los consumidores pagan un depósito por las botellas y latas, el cual les es reembolsado al devolver los envases en máquinas RVM ubicadas en los supermercados. Este sistema ha demostrado ser efectivo para aumentar las tasas de reciclaje y reducir la cantidad de residuos. Sin embargo, se identificó un problema de higiene pública en la ciudad, donde los tachos de basura son vandalizados por recolectores informales que buscan los envases desechados para devolverlos y obtener el depósito. A pesar de esto, el sistema ha logrado fomentar el hábito de reciclaje en los ciudadanos. Se visitó un centro verde, donde se observó la alta tasa de recuperación y reciclaje que no se limita a los envases, sino que se extiende a otros residuos como muebles, ropa, plásticos y maderas en general, entre otros. La separación de residuos es una práctica normal en la rutina de los ciudadanos, quienes acuden a estos centros para entregar sus residuos o muebles y electrodomésticos en desuso. Sin embargo, en el servicio de recolección diferenciada semanal, los residentes de la ciudad reclaman que la frecuencia es insuficiente, lo que lleva a que los contenedores se desborden. Por otro lado, al indagar en el destino final de los materiales reciclables, se informó que la mayoría son procesados en el país, pero se desconoce el destino final de aquellos que son exportados a otros países fuera de Europa. Además, aunque se encontraron numerosas tiendas de productos sostenibles, se admitió que la trazabilidad de los materiales reciclados es limitada, lo que sugiere que la separación de residuos y la venta de productos reciclados son flujos independientes y sin conexión directa.

La última parada del viaje fue Berlín, donde se encontró la cultura de sostenibilidad más avanzada entre las ciudades visitadas. La experiencia comenzó con una reunión con la organización Circular Berlin, una iniciativa que promueve la economía circular en la ciudad a través de la colaboración entre múltiples actores. Su modelo opera como un punto de encuentro que conecta a emprendimientos, empresas, investigadores, ciudadanos y el gobierno local. El objetivo central de la organización es crear una red de actores para compartir conocimientos, debatir ideas e impulsar proyectos de economía circular, con el fin de catalizar el cambio a nivel comunitario y generar un impacto positivo a nivel cultural, económico y ecológico en la ciudad. En este encuentro, un miembro de la organización presentó su misión y destacó proyectos específicos de circularidad que se estaban desarrollando en Berlín, lo que sirvió como una guía estratégica para organizar las visitas a los diferentes puntos de interés durante la estadía. Este encuentro fue un punto de inflexión en el viaje, proporcionando una hoja de ruta para visitar proyectos específicos y comprender la visión integral de la economía circular en la región, mientras que se exploró cómo la colaboración puede ser un motor de cambio.

En la capital alemana, se observó un sistema DRS similar al de Ámsterdam, pero con diferencias importantes. En Berlín, los consumidores también pagan un depósito por las botellas y latas, el cual es reembolsado al devolver los envases en máquinas RVM ubicadas en supermercados. Sin embargo, la conciencia de los ciudadanos ha evolucionado de manera tal que, al desechar envases en la vía pública, prefieren dejarlos junto a los cestos de basura para facilitar que las personas sin hogar los recuperen y obtengan el depósito para adquirir alimentos diariamente con estos fondos. Esta práctica demuestra cómo una iniciativa de reciclaje puede también promover un comportamiento social positivo, evitando los problemas de higiene que se observaron en los Países Bajos. Se visitó un centro verde, similar al de Ámsterdam, donde se reciben y clasifican los residuos. El gran flujo de ciudadanos que acuden a estos centros demuestra el alto compromiso de la población con la reducción y la separación de residuos. De hecho, los trabajadores indicaron que esta es una práctica habitual, ya que los ciudadanos pueden ser multados por dejar grandes volúmenes de residuos en la vía pública. Se identificó, además, una gran presencia de tiendas de segunda mano y comercios circulares que promueven la reutilización y la reducción de residuos. Incluso, se visitó una tienda de productos a granel, donde los clientes llevan sus propios envases reutilizables para comprar alimentos y productos de limpieza. A partir de lo observado, se pudo constatar que la economía circular en Berlín está profundamente integrada en la rutina diaria de sus habitantes.

En retrospectiva, el viaje de investigación a Europa resultó ser una experiencia valiosa. Cada ciudad visitada proporcionó una visión única de los desafíos y éxitos de los sistemas de reciclaje y economía circular. Se puede inferir que el sistema DRS es un método que ha demostrado ser efectivo para fomentar el reciclaje de envases, logrando tasas de recuperación superiores al 90\%, un nivel que no se ha alcanzado con otros enfoques. A su vez, se pudo constatar que la conciencia ciudadana es un factor que influye en el éxito de cualquier sistema de reciclaje. Los modelos de colaboración y fomento de proyectos, como los observados en Berlín, son un vehículo eficaz para impulsar la economía circular a nivel regional. El conocimiento obtenido en este viaje apoya la hipótesis de que una solución de trazabilidad del vidrio puede contribuir no solo a la transparencia de la cadena de valor, sino también a generar conciencia y un impacto positivo en la cultura local.

% TODO: add images. and mention them on the text.