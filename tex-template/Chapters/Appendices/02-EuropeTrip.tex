\chapter{Viaje de Investigación}
\label{cp:europe-trip}

[Esta sección aún está pendiente de redacción]

Aprovechando una oportunidad profesional, se realizó un viaje de investigación de dos semanas a Europa.
Durante este viaje, se visitaron y estudiaron sistemas de reciclaje y programas de incentivos de reciclaje en tres países europeos.
Las actividades incluyeron visitas a centros verdes, interacciones con organizaciones involucradas en el reciclaje, exploración de redes de comercios circulares u orientadas a la sustentabilidad, y encuentros con organizaciones dedicadas a promover la economía circular.
Los hallazgos y observaciones de este viaje, documentados a través de notas y fotografías, se detallan en el Apéndice.

---

En el mes de noviembre de 2024, durante la etapa de definición del alcance de este trabajo, oportuamente (por motivos profesionales ajenos a este trabajo) se realizó un viaje de investigación a Europa para conocer de primera mano los sistemas de reciclaje y economía circular implementados en diferentes países. Los resultados de esta experiencia resultaron ser enriquecedores para este trabajo de tesis, a pesar de haber generado un retraso en su planificación.

Este viaje tuvo como objetivo principal aprender sobre las mejores prácticas en la gestión de residuos y el reciclaje, así como explorar la viabilidad de implementar un sistema similar en ciudades de latinoamérica. A continuación, se detallan los aspectos más relevantes de la experiencia:

- Se visitaron 3 ciudades: Madrid (España), Amsterdam (Países Bajos) y Berlín (Alemania), cada una con sus propios sistemas de reciclaje y gestión de residuos.

- En Madrid, no tienen DRS, se observó la mecánica de recolección diferenciada y las categorías de recolección en origen. Se concluyó que entre los países visitados, España es el que menos ha avanzado en la implementación de sistemas de reciclaje y economía circular. A su vez, cabe destacar que la sustentabilidad y economía circular sí es un tema de interés y preocupación en la sociedad española, pero aún no se han implementado sistemas de recolección diferenciada ni DRS. España ha implementado la Ley REP hace algunos años y en 2025 se comenzó el plan para la implementación de DRS debido a que no alcanzaron sus metas de reciclaje de los últimos años.

- En Amsterdam se usa el sistema DRS, donde los consumidores pagan un depósito por las botellas de vidrio y plástico y latas, que se les devuelve al devolverlas en una RVM, ubicadas en las principales cadenas de supermercados. Con esta técnica se fomenta el hábito de reciclaje de envases en los ciudadanos, al incluirlo como un paso extra al proceso de ir de compras al supermercado. Este sistema ha demostrado ser efectivo para aumentar las tasas de reciclaje y reducir la cantidad de residuos. El problema hallado es que en la ciudad, los tachos de basura son vandalizados por recolectores informales para buscar botellas y latas (quienes usan el dinero obtenido para poder comprar comida diariamente), lo que genera un problema de higiene. Tuvimos la oportunidad de visitar un centro verde, y descubrimos que en esta ciudad las altas tasas de recuperación y reciclaje no se reduce a los envases, sino que también se aplica a otros tipos de residuos como papel, cartón, metales, electrodomésticos y muebles. En este centro verde se reciben los residuos reciclables y se clasifican para su posterior reciclaje. Es lo normal para los cuidadanos dirigirse a estos centros para entregar sus residuos o muebles y electrodomésticos en desuso. Respecto a la recolección diferenciada, se observó que los ciudadanos separan sus residuos en diferentes contenedores ubicados en la vía pública, y la recolección se realiza de manera semanal, de a un material por vez. Cada material se recoge una vez por semana. Los vecinos reclaman que la frecuencia de recolección es insuficiente, ya que los tachos se llenan rápidamente y no hay espacio para más residuos, que quedan expuestos en la vía pública junto a los contenedores. Es notorio que los ciudadanos contemplan el reciclaje como parte de su rutina diaria, y la separación de residuos es una práctica común. Además, se observó que los ciudadanos están muy concientizados sobre la importancia del reciclaje y la reducción de residuos. A su vez, se intentó indagar con el destino final de los residuos reciclables, y se nos informó que la mayoría de los materiales reciclables son procesados en el país, pero que son exportados a otros países fuera de europa para su disposición final y se desconoce el destino final de los mismos o si realmente son reciclados.

- En Alemania


En orden de evolución 
Contar la experiencia de investigación de sistemas de reciclaje en Europa. Contar sobre los DRS y los centros de reciclaje. Adjuntar fotos. Contar cada paso a paso qué cosas hicimos y a qué lugares fuimos. Hacer minuta.

[...continua]

% TODO: finish the europe story and add photos and redact it