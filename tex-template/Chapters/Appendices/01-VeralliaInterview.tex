\chapter{Entrevista a Verallia}
\label{cp:verallia-interview}

\parindent0pt

Para obtener una comprensión profunda de la situación del reciclaje de vidrio y los desafíos que enfrentan los actores de la \gls{cadenadesuministro} en la región, se realizó una entrevista con Lucía J., representante del Área de Medio Ambiente de Verallia, la principal empresa productora de envases de vidrio en la provincia de Mendoza. La entrevista se realizó de forma telefónica, fue semiestructurada y tuvo una duración aproximada de 30 minutos, siendo grabada con el consentimiento previo de la profesional para la utilización de la información en este trabajo de grado.

La conversación se centró en conocer los procesos actuales de producción y reciclaje de vidrio en la empresa, las iniciativas sostenibles que están implementando y las oportunidades para mejorar la \gls{trazabilidad} y la gestión de residuos en su cadena de suministro. Para guiar la conversación, se preparó una lista de preguntas que abarcaban temas como el proceso que realizan, el origen y la clasificación del vidrio reciclado, el rol de sus clientes y proveedores, las reglamentaciones locales y el potencial de un sistema de trazabilidad como el propuesto en esta tesis. Durante la conversación, se permitió una fluidez natural y se profundizó en temas relevantes a medida que la entrevista avanzaba. A continuación, se presenta la transcripción literal de la entrevista, seguida de un análisis y reflexión sobre los puntos más relevantes de la misma.

\section{Transcripción de la entrevista}

\textbf{Entrevistadora:} Primero te pongo un poco en contexto sobre lo que estoy haciendo y después te pido que me cuentes lo que puedas. Como te adelanté, estudio computación. Para mi tesis estoy planteando un sistema para fomentar el reciclaje, especialmente de vidrio, porque encontré que hay muchos sistemas orientados al plástico o residuos electrónicos, pero no tanto al vidrio. Por eso elegí este material para mi trabajo. Por ejemplo, trabajé en una aplicación con la municipalidad de Godoy Cruz, llamada Greeny Points, que incentiva a los usuarios a reciclar. Las personas dejan botellas de plástico o latas en una máquina inteligente, reciben un QR que escanean con el celular y suman puntos, que luego pueden canjear por beneficios como carga en la tarjeta Sube o entradas al cine. Es un sistema que ya probamos y funciona muy bien. Podría ir por algo similar, pero primero necesito entender cómo funciona todo el ecosistema del vidrio. Contame un poco cómo es el proceso que hacen ustedes.

\textbf{Lucía:} Sí, el sistema de reciclado de vidrio. Hoy funciona así: por un lado, hay empresas pequeñas o pymes que recolectan vidrio de la calle, junto con recolectores informales. En algunos municipios hay recolectores en blanco. El problema principal en Mendoza, y en Argentina en general, es que no hay una cultura de reciclaje. 

\textbf{Entrevistadora:} ¿Qué significa esto?

\textbf{Lucía:} Por ejemplo, en algunos municipios te retiran los residuos reciclables ciertos días. Yo vivo en Capital y sé que los martes y jueves hay que sacar papel, cartón, vidrio, todo separado. Pero luego, en el camión, lo mezclan y en la planta recicladora lo separan. En otros países, la gente separa incluso por tipo de vidrio (blanco, verde), porque el sistema lo fomenta. Acá no hay una política general en Mendoza; cada municipio lo hace a su criterio. Tuvimos una reunión con el Ministerio de Medio Ambiente y el director de la fábrica, y hablamos de esto. El problema es que quienes recolectan vidrio informalmente y nos lo venden no tienen la documentación legal habilitada. Como empresa multinacional, necesitamos documentos de AFIP y otros requisitos legales para comprarles, y si no los tienen, no podemos hacerlo. Ellos terminan vendiendo el vidrio en otro lado. Nosotros compramos a quienes sí tienen la documentación, generalmente pymes o empresas chicas. Ese vidrio se limpia porque viene con tapitas, etiquetas, y no todo tipo de vidrio se puede reciclar. Por ejemplo, no podemos reciclar el vidrio marrón de botellas de cerveza, solo blanco y verde. Hay categorías según el diámetro y tipo de envase. Compramos el vidrio, lo almacenamos y lo enviamos a dos empresas que lo limpian y clasifican manualmente, lo que encarece el proceso. Estas empresas siguen ciertos procesos de calidad. Desde abril tenemos una planta de limpieza propia para intentar reciclar más y tener vidrio limpio. El vidrio limpio se mezcla en el horno con la materia prima y se funde para fabricar nuevos envases. 

A nivel de proveedores, la empresa tiene el programa "Vidrio, una acción transparente", que lleva adelante Alejandra Merín, responsable de Responsabilidad Social Empresarial. Este programa realiza acciones externas, como campañas en supermercados y barrios privados, donde hay campanas para que la gente lleve sus botellas. Actualmente tenemos alrededor de 40 campanas ubicadas en diferentes puntos de Capital y Guaymallén. Nosotros las retiramos y las procesamos. Se hacen campañas como el Día del Niño, donde si llevás botellas podés canjearlas por entradas al cine, o sorteos de canastas de mercadería en el Día del Medio Ambiente. Son formas de incentivar el reciclaje, pero el volumen que se recicla por estas campañas es mínimo comparado con lo que necesitamos. Es una forma de fomentar el reciclaje porque no hay políticas provinciales o municipales que acompañen estas acciones. Esa sería la problemática, que no hay políticas que acompañen las iniciativas de reciclaje.

\textbf{Entrevistadora:} Entonces, ¿la mayoría del vidrio que reciben para reciclar proviene de recicladores y solo una mínima parte de las campañas?

\textbf{Lucía:} Exacto, la mayoría llega de pequeñas empresas. También recibimos algo de bodegas, que traen descartes, pero es muy poco. El problema es que comprar vidrio reciclado es caro, más que la materia prima virgen, por la limpieza manual. Por eso tuvimos que bajar los objetivos de reciclaje este año. Si los recolectores informales pudieran vendernos directamente, sería más barato, pero como no tienen la documentación, hay intermediarios y el precio sube. Además, el vidrio tiene que clasificarse bien y a veces se descarta por alto contenido de plomo, porque las botellas de algunas empresas de la competencia que no exportan tienen más plomo. Entonces eso limita la cantidad de vidrio que podemos reciclar porque nosotros exportamos y tenemos que cumplir ciertas normas internacionales, también las bodegas, que son nuestros clientes.

\textbf{Entrevistadora:} ¿Ustedes se encargan de clasificar el vidrio o ya viene clasificado?

\textbf{Lucía:} La clasificación se hace en nuestra planta de limpieza y en las dos empresas externas que limpian el vidrio.

\textbf{Entrevistadora:} ¿Cómo lo clasifican? ¿En qué categorías?

\textbf{Lucía:} Principalmente por color, aunque hay otros parámetros de calidad que desconozco. Separan tapitas, etiquetas y basura, pero detalles específicos los maneja el encargado de la planta.

\textbf{Entrevistadora:} ¿Saben cuánto del vidrio reciclado proviene de sus propios envases y cuánto de otras empresas?

\textbf{Lucía:} No, porque llega vidrio de cualquier lado. Hay competencia en la región y no siempre sabemos el origen exacto de cada botella. Hay otras empresas productoras de envases en Buenos Aires, Santa Fe y San Juan de las que también llegan envases para reciclar, entonces no sabemos de dónde viene cada botella.

\textbf{Entrevistadora:} ¿Considerás que podría ayudar a aliviar el proceso de clasificación implementar una máquina que preclasifique por color o características?

\textbf{Lucía:} Sí, podría ser útil. Si averiguo más sobre la tecnología de clasificación, te voy a avisar.

\textbf{Entrevistadora:} A pesar de haber disminuido la meta anual de reciclaje debido a los costos, ¿Sus proveedores pudieron cumplir con la cantidad de vidrio que necesitaban para reciclar?

\textbf{Lucía:} Sí, hay oferta suficiente de vidrio para reciclar.

\textbf{Entrevistadora:} ¿La restricción de no comprar a recicladores informales es por política de la empresa o legislación local?

\textbf{Lucía:} Es por ambas razones: necesitamos papeles legales y cumplir requisitos impositivos, tanto por la empresa como por la legislación local.

\textbf{Entrevistadora:} Pienso en una aplicación para centralizar la venta de vidrio reciclado mediante una sola empresa y reducir intermediarios y así reducir el precio de adquisición del material. ¿Creés que sería posible?

\textbf{Lucía:} No creo, este proveedor centralizado al mismo precio no lo vendería, aunque la aplicación podría servir para los recicladores informales. Se podría usar una aplicación para que la gente lleve vidrio a puntos verdes, como los que ya hay en plazas de Ciudad y Godoy Cruz. Eso podría aumentar la cantidad de vidrio recolectado. Tenemos muchas campanas distribuidas y la gente puede llevar sus botellas ahí. Para mí, el ciudadano tiene que empezar a tener ese compromiso de separar y llevar el vidrio.

\textbf{Entrevistadora:} También pienso que las bodegas podrían hacer campañas para que los consumidores devuelvan sus botellas a la misma bodega. ¿Creés que podría servir?

\textbf{Lucía:} Sí, sería útil y vendría clasificado. Se podría hacer un convenio para que las bodegas devuelvan botellas propias.

\textbf{Entrevistadora:} ¿Las botellas tienen algún tipo de código para identificarlas?

\textbf{Lucía:} Sí, tienen un cuadradito con una V en la base, antes era un trébol. Algunas dicen "eco" y son más livianas. También tienen un número de molde y puntitos que identifican el modelo.

\textbf{Entrevistadora:} ¿Se podría usar esa información para clasificar automáticamente?

\textbf{Lucía:} Sí, las máquinas podrían identificar esos códigos y ayudar en la clasificación, aunque la limpieza manual siempre será necesaria, que sigue siendo un costo alto.

\textbf{Entrevistadora:} ¿Creés que se podría mejorar el reciclaje asociando lo que venden con lo que reciclan?

\textbf{Lucía:} Sí, podría ayudar a saber cuánto vuelve y mejorar la calidad del reciclado. También podría servir para pagar mejor por el vidrio propio o para hacer los convenios con las bodegas.

\textbf{Entrevistadora:} Por último, ¿la empresa estaría dispuesta a participar en algún experimento o proyecto piloto en el futuro?

\textbf{Lucía:} Sí, todo se puede plantear. Habría que consultarlo a la hora de la gestión, pero sí, todo lo que sume se puede plantear y gestionar.

\textbf{Entrevistadora:} Perfecto, muchas gracias por tu tiempo y toda la información, sirve mucho para mi trabajo.

\textbf{Lucía:} De nada, cualquier otra duda me escribís. ¡Suerte!

\section{Análisis y reflexión}

La entrevista con Lucía proporcionó información valiosa sobre la situación actual del reciclaje de vidrio en Mendoza y los desafíos que enfrenta Verallia en su cadena de suministro. A lo largo de la conversación, se identificaron varias áreas de mejora y oportunidades para implementar un sistema que beneficie a todos los actores involucrados. A continuación, se destacan los puntos más relevantes hablados durante la entrevista:

\begin{itemize}
		\item \textbf{Falta de cultura de reciclaje:} La ausencia de una cultura sólida de reciclaje en Argentina, y específicamente en Mendoza, es un obstáculo significativo. La separación de residuos se limita a iniciativas municipales específicas, y la gente no tiene el hábito de clasificar por tipo de vidrio (blanco, verde, etc.). La falta de políticas municipales coherentes y la dependencia de recolectores informales dificultan la recolección eficiente de vidrio reciclable.

		\item \textbf{Clasificación y limpieza del vidrio:} El vidrio que llega a la planta de reciclaje de Verallia no está clasificado y requiere una limpieza manual, lo que encarece significativamente el proceso. La empresa no puede reciclar cualquier tipo de vidrio (por ejemplo, el vidrio marrón de algunas botellas de cerveza) y se ha visto obligada a bajar sus objetivos de reciclaje por los altos costos operativos, ya que el vidrio reciclado termina representando un costo mayor al vidrio virgen. También hay un problema con el descarte de vidrio que contiene altos niveles de plomo, lo que limita la exportación de envases reciclados. La implementación de tecnologías que faciliten la preclasificación del vidrio podría aliviar la carga de trabajo en las plantas de limpieza y reducir costos.

		\item \textbf{Restricciones legales:} Gran parte del vidrio que llega a Verallia para reciclaje proviene de recolectores informales o pequeñas PyMEs. Lucía menciona que, si bien hay oferta, el proceso está encarecido por la falta de formalidad de estos recolectores, que no cuentan con la documentación legal necesaria para vender directamente a una empresa multinacional como Verallia. Esto obliga a la empresa a comprar a través de intermediarios, lo que aumenta los costos. Un sistema que centralice a los recicladores y garantice la legalidad podría mejorar la eficiencia y reducir costos, reduciendo la dependencia de intermediarios.

		\item \textbf{Iniciativas de concientización:} Las campañas internas y externas para fomentar el reciclaje son valiosas para lograr una mayor concientización ciudadana, pero su impacto en volumen de material reciclado es limitado. Un enfoque más amplio que involucre a clientes y consumidores finales podría aumentar la cantidad de vidrio reciclado.
		
		\item \textbf{Trazabilidad y tecnología:} La posibilidad de implementar un sistema de trazabilidad utilizando códigos o marcas en las botellas presenta una oportunidad para mejorar la gestión del reciclaje. Esto podría facilitar la identificación y clasificación del vidrio, beneficiando tanto a Verallia como a los recicladores.
		
		\item \textbf{Colaboración con clientes:} Involucrar a los clientes, como bodegas, en programas de devolución y reciclaje podría aumentar la cantidad de vidrio recuperado y mejorar la calidad del material reciclado. Lucía reconoce que un sistema de trazabilidad que asocie lo que Verallia vende a las bodegas con lo que después vuelve para reciclar podría ser muy útil. También sugiere que la empresa podría estar dispuesta a participar en un experimento o proyecto piloto si se presenta de forma estructurada.
\end{itemize}
