\chapter{Contenido Externo}
\label{cp:annex-content}

\parindent0pt

A continuación se presentan una serie de recursos adicionales relacionados con el proyecto accesibles en línea.

\section{Demostración}
\label{sec:results-demo}

Como parte de este Trabajo Final se grabó un video demostrativo que resume los aspectos más relevantes del trabajo y presenta el funcionamiento de la plataforma desde la perspectiva del usuario. Este video se encuentra disponible en el siguiente enlace: 
\url{https://github.com/RocioCM/computer-science-thesis/blob/main/docs/system-demo.mov} % TODO: add this link content.

Complementariamente, se documentó el estado final del sistema mediante una serie de capturas de pantalla de la plataforma, comprendiendo las principales funcionalidades de cada pantalla del sistema. Estas capturas pueden consultarse en el siguiente enlace: \url{https://github.com/RocioCM/computer-science-thesis/blob/main/docs/screenshots} % TODO: add this screenshots folder.

A su vez, se encuentra disponible en línea la demostración del prototipo, desplegada en un entorno de pruebas. Esta demostración permite a los usuarios explorar las funcionalidades del sistema en un entorno controlado. La demostración está disponible en el siguiente enlace: \url{https://computer-science-thesis.vercel.app/app}

\section{Código fuente}
\label{sec:source-code}

El código fuente completo del prototipo se encuentra alojado en un repositorio público de GitHub. El repositorio puede ser consultado en el siguiente enlace: \url{https://github.com/RocioCM/computer-science-thesis/tree/main/code}

\section{Documentación técnica}
\label{sec:technical-docs}

La documentación técnica del sistema se encuentra disponible en el repositorio de GitHub mencionado anteriormente. Esta documentación comprende los siguientes aspectos:

\begin{itemize}
	\item Descripción de la arquitectura del sistema y estructura del código.
	\item Detalles sobre la implementación de los contratos inteligentes.
	\item Guía de uso de la API (OpenAPI).
	\item Instrucciones de instalación y despliegue.
	\item Instrucciones para ejecutar las pruebas automatizadas.
	\item Instrucciones para configurar el entorno de desarrollo.
\end{itemize}

\section{Gestión del Proyecto}

El reporte de la planificación y ejecución de las tareas del proyecto, incluyendo la gestión de incidencias y el seguimiento del progreso, se encuentra disponible en la herramienta Jira. Se puede consultar un listado resumido de las incidencias registradas en el siguiente enlace: \url{https://github.com/RocioCM/computer-science-thesis/blob/main/docs/Estado-Jira-Export.pdf}

A su vez, en el siguiente enlace se puede consultar el listado completo de cada una de las tareas e incidencias registradas, incluyendo su descripción completa y metadatos: \url{https://github.com/RocioCM/computer-science-thesis/blob/main/docs/Tareas-Jira-Export.pdf}
