\chapter[Conclusiones]{Conclusiones}
\label{cp:conclusions}

\parindent0pt

En esta sección de conclusiones, se reflexiona sobre el proceso de desarrollo del prototipo de trazabilidad de envases de vidrio, analizando los resultados obtenidos, los desafíos superados y las perspectivas futuras que emanan de este trabajo. Se busca ofrecer una visión completa que no solo se limite a lo técnico, sino que también abarque la experiencia metodológica y el potencial de impacto real de la solución. 

El resultado final de este trabajo fue un prototipo tecnológico funcional basado en blockchain, con una interfaz de usuario y experiencia de usuario pulidas y validado a través de un riguroso proceso de pruebas. A partir de los resultados obtenidos, se considera que el presente trabajo cumple con los objetivos planteados al comienzo y que este prototipo representa una prueba de concepto que demuestra la viabilidad del uso de tecnología blockchain para impulsar la transparencia y la sostenibilidad en la industria vitivinícola de la región.

\section{Análisis de la Metodología}

Al reflexionar sobre el proceso de ejecución de este trabajo y los resultados obtenidos, se considera que la elección del modelo en V para guiar el desarrollo del prototipo fue idónea. Esta metodología, que se asocia habitualmente a proyectos de gran escala que demandan alta calidad, demostró ser altamente eficaz en este contexto. La naturaleza inmutable de la tecnología blockchain, especialmente de los contratos inteligentes, exige un enfoque que minimice la aparición de errores en las etapas finales del ciclo de vida del software. En este sentido, las fases de definición de requerimientos y diseño del software previo a su implementación, así como la ejecución de pruebas unitarias desde la implementación, propuestas por el modelo en V fueron valiosas, permitiendo detectar y corregir inconsistencias de diseño e integración antes de la implementación, lo que facilitó un desarrollo tanto ágil como robusto.

A pesar de la rigidez de la metodología, el proceso de ejecución se adaptó a las circunstancias. Aunque la planificación inicial se desvió, los contratiempos, como los viajes o las cargas académicas, no impidieron el avance continuo del proyecto. De hecho, el viaje de investigación a Europa, aunque desvió el cronograma, enriqueció de forma significativa el proyecto, proporcionando una perspectiva global sobre la economía circular y la cultura del reciclaje. Esta experiencia personal reforzó la convicción sobre la aplicabilidad del trabajo, destacando que la flexibilidad y la resiliencia son también componentes valiosos en la gestión de proyectos académicos.

\section{Reflexiones Finales}

El desarrollo de este trabajo no estuvo exento de desafíos, muchos de los cuales fueron tan importantes como el propio desarrollo del código. El primer gran reto enfrentado fue razonar más allá de la implementación técnica para descubrir las necesidades y dolencias de los actores del ecosistema. Plantear una solución que pudiera atender las particularidades de cada uno, desde el productor hasta el reciclador, requirió una labor de análisis y conceptualización que sentó las bases para el éxito del prototipo.

Desde el punto de vista técnico, la implementación de la blockchain sin experiencia previa representó una curva de aprendizaje considerable. A pesar de su complejidad, se constató que la blockchain es una tecnología sumamente interesante y con un gran potencial para casos de uso que requieren transparencia y descentralización. Los aprendizajes adquiridos a lo largo de este proceso de desarrollo son un activo valioso, y la experiencia con blockchain abre la puerta a futuras investigaciones y aplicaciones relacionadas.

A nivel metodológico, la adaptación de un proceso pensado para un equipo a un trabajo individual fue otro desafío. El uso de herramientas de gestión de proyectos como Jira fue una estrategia eficaz para mantener el orden, la visibilidad del progreso y la trazabilidad de las tareas, indicando que la disciplina metodológica contribuye al éxito incluso en proyectos unipersonales. Por último, la validación del sistema con usuarios en un contexto académico fue un reto que se abordó con creatividad, recurriendo a usuarios voluntarios para simular un entorno de pruebas realista y obtener una retroalimentación valiosa que permitió refinar la interfaz y la experiencia de usuario.

\section{Perspectivas Futuras}

Con una perspectiva a futuro, se considera que este trabajo de grado representa una prueba de concepto con un gran potencial de escalabilidad y expansión. En primer lugar, la arquitectura del sistema podría extenderse para incluir la trazabilidad de otros materiales, como el PET y el aluminio. Además, el prototipo puede servir como el núcleo que potencie una familia de aplicaciones independientes, desarrolladas a la medida de cada actor de la cadena, o que sirvan de incentivo a los ciudadanos, tal como se observó múltiples proyectos revisados en el estado del arte. A su vez, la apertura del sistema es una perspectiva relevante a futuro, ya que la trazabilidad podría iniciarse en cualquier punto de la cadena de valor, para bajar la barrera de ingreso y facilitar la adopción del sistema.

Desde una perspectiva técnica, las mejoras futuras podrían incluir la integración con sensores IoT en las líneas de producción para automatizar la carga de información, minimizando el error humano. A nivel de impacto, la implementación de esta solución podría tener una influencia positiva en la cultura de la sostenibilidad en la región. Como se observó en las investigaciones de campo en Europa, la confianza generada por la trazabilidad logística y la transparencia, puede ser una herramienta para generar conciencia ciudadana. Al proveer a los consumidores de información sobre el ciclo de vida de los productos, se puede impulsar un cambio de comportamiento que beneficie a la economía y ecología locales.

Para finalizar, se considera que este trabajo cumple con los objetivos planteados y hay conformidad con el resultado final, la ejecución de la metodología y los aprendizajes adquiridos. Se considera un logro destacable que el prototipo demuestre la viabilidad para un futuro sistema real que pueda potenciar realmente la transparencia y la sostenibilidad en la industria vitivinícola local, afectando positivamente a la región.
