\chapter[Conclusiones]{Conclusiones}
\label{cp:conclusions}

% TODO: write down refined conclusions here

\parindent0pt

\textit{Resultados de la metodología, contar experiencia de ejecución real vs planeada, dar opiniones de la metodología elegida y si fue idónea para este trabajo. Contar por qué se desvió la planificación y justificar por qué igual es aceptable.}

Fue un trabajo interesante, más de proceso que de resultados. Es un problema acotado, pero sumamente aplicable en el mundo real. Tuvimos contratiempos, se alargó, hubieron viajes en el medio, pero curiosamente relacionados con este trabajo y que supieron aportar perspectiva al problema. Como desafío técnico fue interesante y principalmente pude aprender mucho sobre pruebas de calidad (ya algo de experiencia tenía en planteo de requerimientos y desarrollo de software). Fue interesante la experiencia de trabajar con blockchain, probó ser una tecnología accesible y con valor real, a pesar de sus restricciones que impone al resto del software en general.

\section{Resultados}

El resultado final fue un software testeado y productivo, con una UX y UI pulida. Agregar algunas imágenes y flujos?
Contar algún flujo de la aplicación, por ejemplo el de productor o reciclador.

\section{Desafíos}

Desafíos principales:
\begin{itemize}
	\item Hacer un planteo que pueda resolver las necesidades de todos los actores
	\item Llevar completa la metodología de equipo siendo un equipo unipersonal.
	\item Implementación de la tecnología blockchain con algo de experiencia previa.
	\item Desarrollar el proceso completo de testeos de calidad sin experiencia previa.
	\item Contratiempos de viajes y trabajo y materias.
	\item Resolver metodología de pruebas con usuarios al ser un sistema que involucra muchos actores reales ocupados en sus asuntos.
\end{itemize}

\section{Perspectivas Futuras}

A futuro este sistema puede implementarse realmente y seguir creciendo:

\begin{itemize}
	\item Extender soporte a cadenas de valor de otro tipo de envases: PET, aluminio.
	\item Implementarse como piloto en Mendoza u otras regiones
	\item Nuevos proyectos, como Círculs, pueden nacer inspirados por este trabajo.
	\item El sistema actualmente es cerrado, es decir, sigue desde principio a fin a un mismo grupo de materiales. A futuro se puede extender para comenzar o soltar la trazabilidad de elementos en cualquier paso de la cadena, para bajar la barrera de ingreso.
	\item Se pueden desarrollar distintas aplicaciones independientes que reemplacen a cada módulo del frontend con casos de uso a medida para cada actor.
	\item Se puede integrar el sistema directamente con sensores en la línea de producción de las fábricas para automatizar la carga de información sin posibilidad de error humano de carga de datos.
\end{itemize}
