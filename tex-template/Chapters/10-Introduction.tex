\chapter[Introducción]{Introducción}
\label{cp:introduction}

\parindent0pt

Bienvenido/a a la plantilla \textcolor{maincolor}{\textbf{UMU Tesis}}. Gracias por elegirla para tu tesis, informe u otro proyecto académico. Esta plantilla es una adaptación de la \textbf{\href{https://github.com/joseareia/ipleiria-thesis}{ipleiria-thesis}}, desarrollada originalmente por \textbf{José Areia}, y ha sido modificada específicamente para ajustarse a los requisitos formales y al estilo de presentación de la \textbf{Universidad de Murcia}. Su desarrollo ha requerido unas cuantas horas de trabajo, personalización y aprendizaje, y confío en que su uso te resulte tan útil como lo ha sido para mí el proceso de adaptarla.

\section{Motivación}
Conozco \LaTeX~desde hace varios años, y empecé a utilizarlo por primera vez para redactar mi Trabajo Fin de Máster (TFM). Desde entonces, he seguido utilizándolo en distintos contextos académicos por su potencia, flexibilidad y calidad tipográfica. Sin embargo, al iniciar mi tesis doctoral me encontré con una carencia importante: la Universidad de Murcia no dispone actualmente de una plantilla oficial para la elaboración de tesis doctorales.

Además, la plantilla disponible en Overleaf, que muchos estudiantes suelen encontrar primero, está desactualizada y no cumple del todo con los requisitos actuales ni con la nueva imagen institucional que se espera en la UMU. 

El objetivo de esta adaptación ha sido cubrir esa carencia institucional, ofreciendo una herramienta moderna, bien estructurada y fácil de usar, que permita a los doctorandos de la UMU centrarse en el contenido académico sin preocuparse por el formato. Espero que esta plantilla sea de utilidad para futuros trabajos y contribuya a facilitar el proceso de redacción académica.

\section{Primeros Pasos}

Para comenzar a utilizar esta plantilla, primero necesitas tener conocimientos básicos de \LaTeX. Para ello, puedes consultar el \autoref{cp:latex-tutorial}. Una vez estés familiarizado/a con \LaTeX, podrás optar por trabajar en un entorno local o utilizar un editor en línea.

Si prefieres un editor en línea, te recomiendo encarecidamente \href{https://www.overleaf.com/}{Overleaf}. Aunque Overleaf ofrece una suscripción de pago con tiempos de compilación ampliados, esta plantilla está diseñada específicamente para compilar sin problemas dentro de los límites del plan gratuito. Para usarla en Overleaf, simplemente visita la \href{https://www.overleaf.com/latex/templates/unofficial-polytechnic-university-of-leiria-estg-thesis-slash-report-template/tqgbrncfhwgt}{página oficial de la plantilla} y haz clic en \textit{Use as Template}.

Si optas por utilizar un editor local, primero deberás instalar un sistema \LaTeX~en tu equipo. Existen varias opciones, pero personalmente recomiendo \href{https://www.tug.org/texlive/}{TeX Live} o \href{https://miktex.org/}{MikTeX}. Tras instalarlo, necesitarás elegir un editor de texto para redactar y editar tus documentos. Para ayudarte en esta elección, te recomiendo consultar esta \href{https://tex.stackexchange.com/questions/339/latex-editors-ides}{entrada}, que ofrece una comparativa completa de los distintos editores disponibles.

Una vez tengas todo instalado, puedes clonar o descargar la \href{https://github.com/enriiquee/umu-thesis}{última versión} de la plantilla desde GitHub y empezar a utilizarla.

\begin{block}[warning]
\textit{Esta plantilla ha sido probada exclusivamente en Overleaf. Si decides usarla en un entorno local, lo haces bajo tu propia responsabilidad. No puedo garantizar que funcione correctamente fuera de Overleaf, ya que no he realizado pruebas exhaustivas en instalaciones locales. Si tienes experiencia configurando entornos \LaTeX, siéntete libre de adaptar, modificar o mejorar la plantilla según tus necesidades.}
\end{block}

\section{Obtener Ayuda}

Si estás empezando con \LaTeX~o con esta plantilla, es probable que te encuentres con situaciones en las que no sepas exactamente cómo hacer algo. Cuando surjan dudas, tienes varias opciones. En primer lugar, \href{https://tex.stackexchange.com/}{TeX Stack Exchange} es una comunidad muy activa que puede ayudarte con casi cualquier problema relacionado con \LaTeX. Por supuesto, Google sigue siendo un recurso válido y útil. Si ninguna de estas opciones resuelve tu duda, puedes contactar directamente con el autor original de la plantilla en \textit{\textcolor{blue}{\myuline{enrique.perez2@um.es}}} para temas específicos de la versión original.

\subsection{Errores, Sugerencias y Solicitudes de Funcionalidad}

Si encuentras un error, tienes una sugerencia o deseas solicitar una nueva funcionalidad, puedes hacerlo a través de la pestaña \textit{Issues} del repositorio en \href{https://github.com/enriiquee/umu-thesis}{GitHub}. Intenta ser lo más descriptivo/a posible al reportar el problema e incluye etiquetas adecuadas para facilitar la gestión del informe.

Para sugerencias o nuevas funcionalidades, puedes seguir los pasos anteriores o, si lo prefieres, implementar tú mismo/a el cambio y enviar una \textit{pull request}. 

\subsection{Comentarios, Textos Guía y Advertencias}

Dentro de esta plantilla encontrarás textos informativos bajo el título ``Guía de Redacción''. Estas secciones están pensadas únicamente como orientación para ayudarte a entender qué tipo de contenido debería incluirse en cada parte del documento. No forman parte del código funcional de \LaTeX.

Al explorar los archivos de configuración, notarás que todo está ampliamente comentado. Trabajar con \LaTeX~puede resultar complejo sin documentación adecuada sobre los paquetes utilizados. Por eso, me he esforzado en comentar todos los cambios que he introducido. Es posible que algunos cambios menores no estén comentados, pero los más relevantes o avanzados sí lo están, con explicaciones detalladas.

\section{Nota}

Aunque esta plantilla fue diseñada originalmente para estudiantes del Instituto Politécnico de Leiria, esta  ha sido adaptada para los estudiantes de la UMU. Se puede configurar el idioma \textbf{Español e Inglés}.

Si decides utilizar esta plantilla, te animo a que la menciones en tu trabajo. Para hacerlo, basta con citar la plantilla original usando el comando \verb|\citep{IPLeiriaThesis}| y esta plantilla adaptada \verb|\citep{UMU_Thesis}|. También puedes mostrar tu agradecimiento marcando el repositorio con una \href{https://github.com/enriiquee/umu-thesis}{estrella} en GitHub. 
