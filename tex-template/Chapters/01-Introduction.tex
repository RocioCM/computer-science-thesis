\chapter[Introducción]{Introducción}
\label{cp:introduction}

\parindent0pt

% TODO: rewrite introduction, motivation, structure, etc.

Este capítulo busca introducir cuál es la motivación de este trabajo y su interés, el objetivo que persigue y una breve guía de lectura del documento. 

\begin{block}[todo]
    \textit{Toda esta sección está sin revisar. El contenido sí es correcto, pero tengo que revisar la redacción y el formato.}
\end{block}

\section{Motivación}

La motivación de este trabajo se da principalmente por 3 motivos encadenados:

\begin{enumerate}
	\item El mundo moderno genera más residuos de los que puede gestionar y esto está afectando al medioambiente y a nuestra propia salud.
	\item La economía circular como modelo económico permite mejorar los ciclos de vida de los materiales para reducir residuos y aumentar los beneficios para los distintos actores de la cadena. Actualmente este modelo no se puede implementar funcionalmente debido a deficiencias de trazabilidad y falta de transparencia a lo largo de la cadena.
	\item Recientemente la tecnología blockchain se está posicionando como una tecnología útil y confiable para brindar trazabilidad y confiabilidad a sistemas con múltiples actores y con falta de confianza entre ellos.
\end{enumerate}

Por último, de entre todos los residuos posibles elegimos:

\begin{itemize}
	\item \textbf{Vidrio}: su alta reciclabilidad y bajo costo de procesamiento \cite{prodvidrio2024verallia} presentan una oportunidad perfecta para la trazabilidad efectiva en la cadena de suministro.
	\item \textbf{Botellas}: por ser Mendoza una provincia productora y exportadora de vino embotellado en botellas de vidrio, se considera un material de interés local que se produce en volumen en la provincia.
\end{itemize}

\section{Objetivos}

\begin{itemize}
	\item Entender los procesos de adopción de tecnologías tales como blockchain y las capacidades actuales en la región para el uso de sistemas de trazabilidad.
	\item En todo lo referido a las Ciencias de la Computación, se busca que la alumna pueda desarrollar una aplicación prototipo funcional basada en tecnología blockchain. Esto permitirá la trazabilidad transparente, segura y en tiempo real de la gestión de residuos, en particular el vidrio, desde su generación hasta su disposición final, con el fin de garantizar el cumplimiento normativo, mejorar la eficiencia operativa y aumentar la confianza entre todos los actores involucrados en el proceso.
\end{itemize}

\section{Estructura general del documento}


\begin{block}[todo]
    \textit{Esto lo voy a redactar al final recién, cuando tenga el documento completo.}
\end{block}

Este documento se estructura en las siguientes secciones [...]. Primero se introducen conceptos base sobre el problema y la tecnología, luego se indaga en soluciones existentes y antecedentes académicos. Después de entrar en contexto, se explica la metodología de trabajo elegida y se describe la ejecución del diseño, desarrollo y pruebas de la solución. Por último, se listan conclusiones y perspectivas futuras. Al final del trabajo se pueden encontrar anexos como lectura opcional y un glosario de términos específicos del problema o la tecnología que puede resultar de utilidad.
