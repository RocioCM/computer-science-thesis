\chapter[Modelado de Requerimientos]{Modelado de Requerimientos}
\label{cp:modelling}

\parindent0pt

-----

# Análisis de requerimientos

La introducción de este documento tiene como objetivo proporcionar un panorama global del proyecto, estableciendo claramente el propósito, el alcance, el valor y el público objetivo del producto. El objetivo final de este documento es definir de forma exhaustiva los requerimientos funcionales y no funcionales del sistema prototipo de trazabilidad de envases de vidrio, así como los requerimientos específicos del frontend del sistema. Estos requerimientos servirán como base para el diseño, desarrollo y validación del sistema.

#### Propósito del Producto

El propósito de este sistema es mejorar la trazabilidad en los modelos de economía circular orientados al reciclado de vidrio. Al proporcionar una plataforma que registre y audite las transacciones de residuos de vidrio a lo largo de toda la cadena de suministros, el sistema busca garantizar la transparencia y la eficiencia desde la producción del envase hasta su disposición final. Este registro confiable y verificable facilitará el cumplimiento normativo y mejorará la gestión de recursos en el sector.

#### Alcance del Producto

El alcance de este proyecto incluye el desarrollo de un contrato inteligente en la blockchain de Ethereum, una interfaz web para interactuar con dicho contrato, una base de datos SQL para almacenar información adicional, y una API que conecte todos estos componentes, junto con una API pública para facilitar la integración con otros sistemas. Además, se desarrollará un frontend prototipo específico para cada actor de la cadena de suministro de envases de vidrio (productores, comerciantes, recicladores y consumidores) que interactuará con el sistema de trazabilidad. El objetivo es centralizar y simplificar el acceso a la información de trazabilidad para todos los actores de la cadena de suministro de envases de vidrio, incluyendo productores, comerciantes, consumidores y recicladores.

#### Valor del Producto

El valor principal del sistema reside en su capacidad para garantizar la integridad y la transparencia de la información a lo largo de la cadena de suministro del vidrio reciclado. Este sistema no solo ayuda a cumplir con las normativas internacionales y locales, sino que también ofrece un beneficio competitivo a las empresas al mejorar la percepción de marca y aumentar la confianza del consumidor en productos sostenibles y responsables. Además, facilita la adopción de prácticas más sostenibles en la industria del vidrio al disponibilizar información detallada sobre la trazabilidad y el impacto ambiental de los envases.

#### Público Objetivo

El público objetivo del sistema incluye a todos los actores involucrados en la cadena de suministro de envases de vidrio, desde productores que necesitan certificar la sostenibilidad de sus productos hasta recicladores interesados en optimizar el proceso de recolección y reutilización de materiales. Además, reguladores y organismos de certificación podrán utilizar el sistema para verificar el cumplimiento de las normativas pertinentes.

#### Uso Previsto

El sistema será utilizado de diversas maneras según el rol del usuario:

- **Productores** podrán registrar y verificar la producción de envases de vidrio con materias primas vírgenes o recicladas.
- **Comerciantes** utilizarán el sistema para gestionar compras y verificar el origen del vidrio.
- **Recicladores** podrán documentar el proceso de reciclaje y su cumplimiento con las normativas.
- **Consumidores** tendrán acceso a información sobre la trazabilidad y sostenibilidad de los productos que compran y que reciclan.

Cada uno de estos roles contará con interfaces y funcionalidades específicas diseñadas para facilitar estas operaciones. El sistema se convertirá en una fuente de información confiable y verificable para todos los actores involucrados en la cadena de suministro de envases de vidrio, a partir del cual se puedan desarrollar aplicaciones y servicios adicionales.

## Requerimientos

Un requerimiento de sistema representa una necesidad específica que el sistema de software debe satisfacer para cumplir con sus objetivos. Los requerimientos son fundamentales porque definen qué debe hacer el sistema (funcionalidad) y cómo debe comportarse (calidad y restricciones). Los requerimientos se clasifican en dos categorías:

- **Requerimientos Funcionales**: Describen las funcionalidades que el sistema debe proporcionar. Estos requerimientos detallan las acciones que el sistema debe ser capaz de realizar, los procesos que debe soportar y las interacciones que debe permitir entre los usuarios y el sistema.

- **Requerimientos No Funcionales**: Establecen las características de calidad que debe cumplir el sistema, como rendimiento, seguridad, escalabilidad y usabilidad. Estos no se centran en las funcionalidades específicas, sino en cómo el sistema debe ejecutarlas.

Cada requerimiento debe ser claro, específico, medible y verificable. Un requerimiento se especifica generalmente en un formato que incluye un identificador único y una descripción detallada. A continuación, se presentan los requerimientos funcionales y no funcionales del sistema prototipo de trazabilidad de envases de vidrio, así como los requerimientos específicos del frontend del sistema.

------

El modelado de requerimientos constituye la etapa inicial del lado izquierdo del modelo en V, que enfatiza la importancia de las pruebas en cada fase del ciclo de vida del proyecto de software. El objetivo principal de esta etapa es comprender, documentar y validar las necesidades y expectativas de los interesados del sistema, definiendo de forma precisa su comportamiento y funcionalidades. Esta fase se asocia directamente con las pruebas de aceptación del lado derecho de la V, la etapa final del modelo, en la cual se verifica que el sistema cumple con los requerimientos definidos inicialmente.

Un modelado de requerimientos preciso incide en las etapas subsiguientes de diseño, implementación y pruebas. Los errores o ambigüedades en la fase de modelado pueden propagarse a lo largo del proyecto, resultando en un aumento del tiempo y los recursos requeridos para corregirlos. Por lo tanto, la inversión de esfuerzo para asegurar que los requerimientos sean claros, completos y factibles reduce el riesgo de inconsistencias y la necesidad de refactorizaciones en fases posteriores, contribuyendo a la ejecución exitosa del proyecto.

En el contexto de este trabajo, centrado en el desarrollo de un prototipo de aplicación con tecnología blockchain para la trazabilidad y valorización de envases de vidrio, el modelado de requerimientos se ejecutó de forma estructurada para garantizar que el prototipo respondiera a las necesidades específicas de una economía circular sostenible y transparente.

El proceso de modelado de requerimientos se estructuró en una serie de pasos iterativos para descubrir, definir y refinar los requisitos del sistema. Las etapas del proceso se resumen en la Figura \ref{fig:requirements-modelling-process}.

\begin{figure}[!htpb]
    \centering
    \includegraphics[width=0.6\textwidth]{Figures/requirements-modelling.png}
    \caption{Etapas del proceso de modelado de requerimientos del prototipo de trazabilidad de vidrio}
    \label{fig:requirements-modelling-process}
\end{figure}

El proceso dio inicio con una investigación exhaustiva del dominio del problema para identificar los actores clave y sus interacciones (Sección \ref{sec:domain-definition}). A partir de esta información, se elaboró un \textit{Canvas de Propuesta de Valor} para documentar de manera flexible las necesidades y problemáticas de cada actor. Posteriormente, se modelaron los casos de uso para describir las funcionalidades del sistema desde la perspectiva del usuario (Sección \ref{sec:use-cases}). Estos casos de uso constituyeron la base para la definición formal de los requerimientos funcionales y no funcionales, incluyendo sus interdependencias (Sección \ref{sec:requirements-definition}). Finalmente, se redactaron \textit{historias de usuario} con un nivel de detalle suficiente para definir los criterios de aceptación, lo cual permitió iniciar las etapas de diseño de arquitectura, estimación de esfuerzo y planificación de la implementación.

\section{Definición de Dominio}
\label{sec:domain-definition}

El modelado de requerimientos inicia con la definición del dominio del problema. En el contexto de este proyecto, que busca la trazabilidad y valorización del vidrio, el objetivo es comprender el entorno en el que el sistema operará, identificando los actores y sus interacciones. El análisis establece la base para la construcción de los requerimientos del sistema.

La definición del dominio se realizó a través de una investigación y revisión de la literatura sobre la trazabilidad del vidrio. En la Sección \ref{sec:related-work}, se exploraron los trabajos existentes en el área de blockchain aplicada para lograr una economía circular, también se investigaron proyectos existentes que aborden la misma temática o el uso de tecnología para el mismo fin. A su vez, se realizaron entrevistas e investigaciones de campo con expertos en la industria del vidrio y el reciclaje regional, para comprender los procesos actuales, las problemáticas y las expectativas de los actores involucrados (Apéndices \ref{cp:verallia-interview} y \ref{cp:europe-trip}). Se examinaron las etapas del ciclo de vida de los envases de vidrio, desde su producción hasta su reintroducción en la cadena de valor, para identificar los puntos donde la tecnología blockchain puede aplicarse. La Figura \ref{fig:glass-lifecycle} ilustra el ciclo de vida de los envases de vidrio y los actores involucrados en cada etapa.

\begin{figure}[!htpb]
    \centering
    \includegraphics[width=0.6\textwidth]{Figures/glass-lifecycle.png}
    \caption{Etapas del ciclo de vida de los envases de vidrio}
    \label{fig:glass-lifecycle}
\end{figure}

En este marco, se identificaron los siguientes actores clave:

\begin{itemize}
    \item \textbf{Productor de Vidrio (Productor Primario):} Fabricante del envase de vidrio, con la responsabilidad de registrar la información inicial del lote de material.
    \item \textbf{Productor de vino (Productor Secundario):} Empresa productora de vino que utiliza el envase de vidrio para embotellar sus productos y requiere acceder a la información de trazabilidad con el fin de garantizar la calidad y seguridad alimentaria.
    \item \textbf{Consumidor:} Usuario final que adquiere bebidas envasadas, las consume y puede participar en el proceso de reciclaje.
    \item \textbf{Centro de Reciclaje:} Entidad que recibe, procesa y recicla el vidrio, utiliza la información de trazabilidad para verificar la calidad del material reciclado y dejar registro de su disposición final.
\end{itemize}

Tras la identificación de los actores, la revisión de la literatura y la investigación de campo y entrevistas, se elaboró un Canvas de Propuesta de Valor. Este diagrama semi-estructurado es una herramienta estratégica que documenta de manera flexible las necesidades y desafíos de cada actor, facilitando la comprensión de sus expectativas.

El canvas se divide en dos secciones principales: el perfil del actor y el mapa de valor de la solución. En el perfil del actor, se detallan sus necesidades, deseos y miedos, lo que proporciona una visión clara de sus motivaciones y los obstáculos que enfrentan en su estado actual. En el mapa de valor de la solución, se definen las funcionalidades y experiencia que ofrecerá la solución, con el objetivo de aplacar los miedos y satisfacer las necesidades y los deseos de los actores. Este análisis conjunto de las expectativas y la solución permite adoptar un enfoque centrado en el usuario desde el inicio del proceso de diseño de solución, asegurando que el sistema propuesto aborde de manera efectiva los problemas y oportunidades identificados.

La Figura \ref{fig:value-proposition-canvas} muestra el canvas elaborado para este sistema de trazabilidad, detallando las necesidades y expectativas de cada actor. Este análisis, al contrastar las necesidades y expectativas de los actores con las funcionalidades de la solución, sirve como el primer paso para conceptualizar cómo el sistema de trazabilidad basado en blockchain puede mitigar los problemas existentes en la cadena de valor de los envases de vidrio y generar valor tangible para cada participante.

\begin{figure}[!htpb]
    \centering
    \includegraphics[width=0.8\textwidth]{Figures/value-proposition-canvas.png}
    \caption{Canvas de Propuesta de Valor para el sistema de trazabilidad de vidrio}
    \label{fig:value-proposition-canvas}
\end{figure}

El análisis detallado a través del Canvas de Propuesta de Valor revela que los principales desafíos se centran en la falta de transparencia y la ineficiencia de los procesos actuales involucrados en el ciclo de vida del vidrio. El Productor Primario necesita un flujo constante de materia prima reciclada de calidad para minimizar sus costos y cumplir con metas de sostenibilidad. A su vez, el Productor Secundario enfrenta el desafío de verificar la procedencia de los envases de vidrio para garantizar la seguridad alimentaria y, de esta manera, acceder a mercados regulados y sostenibles, cumpliendo sus propias metas de sostenibilidad. Por su parte, el Consumidor busca una forma simple de reciclar, con la seguridad de que su esfuerzo es valorado y recompensado, y desea tener la capacidad de conocer el impacto ambiental de los productos que adquiere. Finalmente, el Centro de Reciclaje se enfrenta a altos costos operativos y a falta de calidad en el material recibido, lo que dificulta su valorización. Estas problemáticas y expectativas compartidas se traducen en la necesidad de un sistema unificado y confiable, que permita a los actores verificar la procedencia del material, documentar cada etapa del ciclo de vida y ofrecer incentivos a los usuarios finales. Por lo tanto, los casos de uso del sistema se diseñan para abordar directamente estas necesidades, permitiendo el registro de lotes de producción de envases, consulta de la trazabilidad de cada unidad, gestión del reciclaje de los envases y verificación de la calidad del material reciclado.

% TODO: acá poner las soluciones también, son como una previa a los casos de uso.

La información recopilada en esta fase proporciona una comprensión de las necesidades de los actores y sienta las bases para la siguiente etapa del modelado de requerimientos, donde se definirán los casos de uso del sistema con una perspectiva centrada en el usuario.

\section{Modelado de Casos de Uso}
\label{sec:use-cases}

Después de definir el dominio y los actores, se procede a la identificación y modelado de los casos de uso. Un caso de uso describe una interacción atómica entre un actor y el sistema para alcanzar un objetivo específico, representando una funcionalidad desde la perspectiva del usuario.

Con el objetivo de acotar el alcance de este trabajo, se realiza el modelado de los casos de uso relacionados con la trazabilidad de los envases de vidrio, abarcando las acciones que los actores pueden realizar en el sistema. Estos casos de uso se centran en las funcionalidades esenciales que permiten a los actores registrar, consultar y gestionar la información relacionada con los envases de vidrio a lo largo de su ciclo de vida. Los casos de uso relacionados con la certificación de cada etapa del proceso e integración con sistemas preexistentes se consideran fuera del alcance de este trabajo, pero los casos de uso modelados permiten sentar las bases para futuras extensiones del sistema hacia estas funcionalidades.

En el prototipo de sistema de trazabilidad para envases de vidrio, los casos de uso comprenden tanto las acciones inherentes al ciclo de vida del material como las interacciones propias de una plataforma digital basada en blockchain. Los primeros tipos de caso de uso describen las operaciones fundamentales del proceso de trazabilidad, como el registro de un nuevo lote de envases fabricado por parte del Productor Primario, la consulta del origen del envase por parte del Consumidor y la recepción de envases reciclables por el Centro de Reciclaje. Los segundos tipos de caso de uso, por su parte, abarcan las funcionalidades básicas del sistema, tales como la autenticación de usuarios y la visualización de datos en la interfaz.

Los casos de uso se representan gráficamente a través de un diagrama de casos de uso que muestra las interacciones entre los actores y el sistema. La Figura \ref{fig:use-case-diagram} presenta el diagrama elaborado para el sistema de trazabilidad de envases de vidrio para una economía circular, ilustrando las funcionalidades mínimas que debe implementar el sistema y los actores asociados a cada funcionalidad. Cada caso de uso está vinculado a un actor específico, lo que permite visualizar claramente las interacciones y responsabilidades de cada uno. En caso de que un caso de uso sea compartido por varios actores, se puede representar como un caso de uso heredado, indicando que varios actores pueden realizar la misma acción o funcionalidad.

\begin{figure}[!htpb]
    \centering
    \includegraphics[width=0.8\textwidth]{Figures/use-case-diagram.png}
    \caption{Diagrama de Casos de Uso del sistema de trazabilidad de vidrio}
    \label{fig:use-case-diagram}
\end{figure}

El diagrama muestra que cada actor tiene un conjunto de casos de uso alineado con sus responsabilidades. Por ejemplo, el Productor Primario puede registrar un nuevo lote de envases producidos o registrar la venta de envases a un Productor Secundario, mientras que el Productor Secundario puede consultar la trazabilidad de origen de los envases recibidos y registrar su uso asociando los envases a un lote de producto final. Por otro lado, el Consumidor puede consultar el origen de un envase adquirido y el destino de un envase enviado a reciclaje, mientras que el Centro de Reciclaje puede recibir envases reciclables, consultar su composición de materiales y registrar su reciclaje. Los casos de uso de los diferentes actores se interrelacionan de manera cíclica, integrando la información de cada actor de la cadena en un único sistema de información interrelacionado que refleja el flujo de la economía circular de envases de vidrio. Por ejemplo, la acción "Vender envases a productores secundarios" del Productor Primario está ligada a los casos de uso del Productor Secundario, permitiendo rastrear el vidrio a lo largo de su ciclo de vida.

La lista de casos de uso sirve como punto de partida para la definición de los requerimientos funcionales y no funcionales del sistema, donde cada caso de uso se descompone en uno o más requerimientos específicos que describen las funcionalidades que el sistema debe implementar para cumplir con las expectativas de los actores.

\section{Definición de Requerimientos}
\label{sec:requirements-definition}

Después de identificar los casos de uso, se definen los requerimientos funcionales y no funcionales del sistema. Los requerimientos funcionales describen las funcionalidades específicas que el sistema debe ofrecer a los usuarios, mientras que los requerimientos no funcionales establecen las características de calidad que el sistema debe cumplir, tales como rendimiento, seguridad o usabilidad.

Los requerimientos se documentan de manera estructurada, asignando un identificador único a cada requerimiento para su seguimiento durante las etapas posteriores de diseño, implementación y pruebas del sistema. La descripción de cada requerimiento incluye su propósito, las condiciones bajo las cuales se cumple y las dependencias con otros requerimientos. Un ejemplo de requerimiento funcional es que "el sistema debe permitir al Productor de Vidrio registrar un nuevo lote de vidrio, especificando la cantidad y el tipo de vidrio", mientras que un requerimiento no funcional puede ser que "el sistema debe garantizar la seguridad de los datos del usuario mediante autenticación y autorización".

A partir de los casos de uso previamente identificados, se definieron 28 requerimientos funcionales y 6 requerimientos no funcionales para el sistema. Estos requerimientos se documentan en la etapa de modelado para su posterior seguimiento durante el desarrollo. Se establecen dependencias entre ellos, lo que permite identificar restricciones en el orden de implementación de las funcionalidades del sistema. Por ejemplo, el registro de un lote de vidrio (RF-006) es una dependencia para que dicho lote pueda ser recibido por un productor secundario para envasar productos (RF-016). La Tabla \ref{tab:functional-requirements} presenta los requerimientos funcionales, mientras que la Tabla \ref{tab:non-functional-requirements} muestra los requerimientos no funcionales definidos para el prototipo de trazabilidad de envases de vidrio basado en blockchain.

\begin{xltabular}{\textwidth}{@{} L{1.5cm} L{2.5cm} Y L{1.5cm} @{}}
	\caption{Requerimientos Funcionales del sistema de trazabilidad de envases de vidrio}
	\label{tab:functional-requirements}\\
	\toprule
	ID & Título & Descripción & Deps \\
	\midrule
\endfirsthead

\toprule
ID & Título & Descripción & Deps \\
\midrule
\endhead

\midrule
\multicolumn{4}{r}{\footnotesize Continúa en la siguiente página}
\\\bottomrule
\endfoot

\bottomrule
\endlastfoot
	RF-001 & Registrar usuarios & El sistema debe permitir registrar usuarios mediante correo electrónico u otro medio con diferentes roles para hacer uso de las distintas partes del sistema. Los roles disponibles en esta primera etapa son: Productor Primario, Productor Secundario, Consumidor y Reciclador. & - \\
	RF-002 & Ingresar a la plataforma & Todos los usuarios deben poder ingresar a la plataforma con su correo electrónico registrado mediante algún método de autenticación con contraseña, OTP, 3rd party, etc. & RF-001 \\
	RF-003 & Mantener sesión de usuario & Cada cuenta de usuarios debe poder mantener abiertas múltiples sesiones en simultaneo. El usuario debe cerrar una sesión individual en cualquier dispositivo. La sesión debe mantenerse abierta en el dispositivo a lo largo del tiempo a pesar de que se cierre el navegador o aplicación. & RF-002 \\
	RF-004 & Validar Autorización & Cada rol de usuario debe tener ciertos permisos y un usuario con un rol dado no debe poder realizar acciones que requieran un permiso del que no goza. Detalle:\n Productor: CRUD productos.\n Comerciante: RU productos (transferencia de propiedad de productos)\n Consumidor: R productos (consultar composición de productos)\n Reciclador: R productos, CRUD lotes de reciclaje. \n Invitados (sin autenticación): permiso de lectura en todo el sistema (R productos, R lotes de reciclaje). & RF-002 \\
	RF-005 & Ver mi perfil de usuario & Cada usuario debe poder consultar la información personal asociada a su cuenta y modificar algunos datos: Email, Nombre Empresa/persona (modificable), Responsable (modificable), Nro teléfono (modificable), Dirección pública blockchain. & RF-002 \\
	RF-006 & Cargar lotes de producción & El productor puede cargar lotes de producción de envases de vidrio incluyendo la siguiente información: Cantidad de envases, Peso por envase, Color, Composición, Espesor, entre otras. & RF-004 \\
	RF-007 & Editar lotes de producción & El productor puede editar la información de producción en caso de equivocación hasta antes de comercializar el lote. & RF-006 \\
	RF-008 &Eliminar lotes de producción &El productor puede eliminar un lote en caso de equivocación hasta antes de comercializar el lote. & RF-006 \\
	RF-009 & Consultar historial de producción & El productor puede consultar información histórica de todos los lotes que ha creado con sus detalles y puede consultar su trazabilidad posterior. & RF-006 \\
	RF-010 & Consultar material reciclable ingresado & El productor puede ver la lista de los lotes o conjuntos de materiales reciclables que volvieron a ingresar a su fábrica (como compra a la recicladora o devoluciones desde bodegas). & RF-006, RF-018, RF-025 \\
	RF-011 & Sacar de circulación envases & El productor puede marcar grupos de botellas de un lote como fuera de circulación o enviado a reciclar (en caso de rotura, falla o desaparición). & RF-006 \\
	RF-012 & Vender envases a productores secundarios & El productor puede marcar cierta cantidad de envases del lote como vendidos a un productor secundario específico. Los envases dejan de ser propiedad del productor y ya no puede modificarlos ni revenderlos. & RF-006 \\
	RF-013 & Asociar envase a código comercial & El productor secundario puede asociar un código de barras (u otro tipo de código visible en la etiqueta impresa de su producto, como QR) a conjuntos de botellas compradas. EL CÓDIGO DEBE GARANTIZAR UNICIDAD. & RF-012 \\
	RF-014 & Modificar código comercial & El productor secundario puede editar la información de códigos asociados a envases hasta antes de vender el producto. & RF-013 \\
	RF-015 & Eliminar código comercial & El productor secundario puede eliminar códigos asociados a envases hasta antes de vender el producto. & RF-013 \\
	RF-016 & Consultar inventario de envases nuevos & El productor secundario puede ver la lista de conjuntos de envases que ha comprado pero aún no ha utilizado (no han sido asociados a ningún producto propio ni código). El productor puede confirmar conformidad o rechazar la compra. & RF-012 \\
	RF-017 & Rechazar envases recibidos & El productor secundario puede desconocer la transacción de transferencia de envases desde el productor en caso de no reconocer la compra o realizar una devolución por algún tipo de error. & RF-012 \\
	RF-018 & Sacar de circulación envases & El productor secundario puede devolver botellas en caso de no conformidad, fallas de fábrica o rotura. Estos envases pueden transferirse al productor como material reciclable o descartarse. & RF-012 \\
	RF-019 & Consultar historial de producción & El productor secundario puede consultar el historial de sus productos embotellados o botellas utilizadas. A su vez puede consultar su trazabilidad posterior a la comercialización. & RF-013 \\
	RF-020 & Vender envases a comerciantes & El productor secundario puede marcar sus productos embotellados como vendidos al comerciante. & RF-013 \\
	RF-021 & Consultar la historia de un envase & Mediante el código asociado a la botella, el ciudadano debe poder consultar el origen, composición y trazabilidad histórica de envase cualesquiera. & RF-013 \\
	RF-022 & Marcar envase como reciclable & El ciudadano puede registrar un envase como ingresado al sistema de reciclaje escaneando su código. & RF-013 \\
	RF-023 & Dar seguimiento a envases & El ciudadano puede hacer seguimiento del destino y trazabilidad hasta la disposición final de todos los envases que ingresó al sistema de reciclaje. & RF-022 \\
	RF-024 & Consultar información del envase para clasificación & El reciclador clasificador puede escanear el código de la botella y obtener información relevante sobre su composición para su correcta clasificación. & RF-013 \\
	RF-025 & Crear lotes de material reciclado & El reciclador puede crear lotes de material reciclado a partir de un conjunto de envases reciclables recibidos de los ciudadanos. Cada lote tiene los siguientes atributos: Peso, Dimensión (si aplica), Material, Composición. & RF-022 \\
	RF-026 & Editar lote de material reciclado & El reciclador puede editar la información de un lote en caso de equivocación hasta antes de su comercialización. & RF-025 \\
	RF-027 & Eliminar lote de material reciclado & El reciclador puede eliminar un lote en caso de equivocación hasta antes de su comercialización. & RF-025 \\
	RF-028 & Vender lote de material reciclado & El reciclador puede marcar un lote de material reciclable como vendido a un productor y debe especificar el comprador. Se asume en este caso que el material fue efectivamente reciclado, finalizando la trazabilidad. & RF-025 \\
\end{xltabular}

\begin{xltabular}{\textwidth}{@{} L{1.5cm} L{2.5cm} Y @{}}
	\caption{Requerimientos No Funcionales del sistema de trazabilidad de envases de vidrio}
	\label{tab:non-functional-requirements}\\
	\toprule
	ID & Título & Descripción \\
	\midrule
\endfirsthead

\toprule
ID & Título & Descripción \\
\midrule
\endhead

\midrule
\multicolumn{4}{r}{\footnotesize Continúa en la siguiente página}
\\\bottomrule
\endfoot

\bottomrule
\endlastfoot
RNF-01 & Transparencia & La trazabilidad de un producto debe ser libremente accesible por cualquier usuario autenticado del sistema en todo momento. \\
RNF-02 & Disponibilidad & El sistema debe estar disponible para su uso 24/7 \\
RNF-03 & Escalabilidad & El sistema debe soportar un nro creciente de transacciones \\
RNF-04 & Mantenibilidad & El sistema debe poder ser mantenible por otros desarrolladores de la industria actual en un futuro \\
RNF-05 & Interoperabilidad & El sistema debe ser integrable con otros múltiples sistemas de stock y gestión de terceros preexistentes \\
RNF-06 & Integridad & Los datos de trazabilidad no deben poder ser alterados luego de cargados sin dejar registro público de la modificación \\
\end{xltabular}

La lista de requerimientos funcionales y no funcionales sirve como base para la siguiente fase de modelado, en la que se definirán las historias de usuario y se planificará el desarrollo del sistema. A su vez, la lista de requerimientos se utilizará posteriormente en la validación y verificación del sistema, asegurando que todas las funcionalidades implementadas cumplan con las expectativas y necesidades de los usuarios.

\section{Historias de Usuario y Planificación}
\label{sec:user-stories}

A partir de la definición de los requerimientos funcionales y no funcionales del sistema, se procede a la creación de las historias de usuario. Las historias de usuario permiten documentar las funcionalidades del sistema desde la perspectiva de sus actores, utilizando un formato estandarizado que describe el rol, la acción deseada y el beneficio esperado: "Como [rol], quiero [acción], para [beneficio]". Por ejemplo, "Como Productor Primario,  
quiero poder editar la información de un lote de envases antes de su comercialización, para poder corregir cualquier error en los datos de producción y asegurar la precisión en el registro". Esta forma de documentar requerimientos facilita la priorización de funcionalidades y la comprensión de las necesidades desde un enfoque centrado en el usuario a la hora de desarrollar el sistema.

Cada historia de usuario se complementa con criterios de aceptación que establecen las condiciones necesarias para su validación, vinculándose directamente con uno o más requerimientos previamente definidos. Esta trazabilidad entre los requerimientos y las historias de usuario es un mecanismo de control que guía el proceso de desarrollo y asegura que la implementación cumpla con las expectativas planteadas. En el contexto del modelo en V, las historias de usuario establecen la base para la fase de pruebas de aceptación, garantizando que el sistema final se alinee con los objetivos del proyecto.

A continuación se presenta un ejemplo de historia de usuario con sus respectivos criterios de aceptación, que ilustra la relación entre las funcionalidades y las necesidades de los actores.

\begin{center}
\fbox{
  \begin{minipage}{0.95\linewidth}
    \textbf{Historia de Usuario:} Consultar historial de producción de lotes de envases de vidrio \\
    \textbf{Como} productor, \\
    \textbf{Quiero} poder consultar el historial de todos los lotes de producción con sus detalles y trazabilidad, \\
    \textbf{Para} poder revisar la información de producción y rastrear cada lote en su ciclo de vida.
    
    \vspace{0.5cm}
    
    \textbf{Criterios de Aceptación:}
    \begin{enumerate}
      \item \textbf{Visualización del historial de lotes:}
      \begin{itemize}
        \item El sistema debe mostrar una lista de todos los lotes creados por el productor con la siguiente información:
        \begin{itemize}
          \item \textbf{Código de lote}
          \item \textbf{Fecha de producción}
          \item Cantidad de envases
          \item Peso por envase
          \item Color
        \end{itemize}
      \end{itemize}
      
      \item \textbf{Acceso a detalles de cada lote:}
      \begin{itemize}
        \item Al seleccionar un lote específico, el sistema debe mostrar los detalles completos, incluyendo:
        \begin{itemize}
          \item Espesor
          \item \textbf{Fecha de producción}
          \item Observaciones adicionales
        \end{itemize}
      \end{itemize}
    \end{enumerate}
  \end{minipage}
}
\end{center}

En el presente trabajo, se definieron un total de 28 historias de usuario, donde cada historia de usuario se corresponde exactamente con un requerimiento funcional del sistema. Los requerimientos no funcionales se abordan de manera transversal, asegurando que aspectos como el rendimiento, la seguridad y la usabilidad sean considerados en el diseño e implementación del sistema.

Para la gestión del proceso de desarrollo, las historias de usuario se registraron en la herramienta Jira, un software para gestión de proyectos de desarrollo de software compatible con la metodología Kanban. Esta herramienta permite gestionar el flujo de trabajo, asignar tareas a los miembros del equipo y realizar un seguimiento del progreso durante el desarrollo. Como parte del proceso de planificación, se estimó el esfuerzo necesario para realizar cada tarea y se registró en la herramienta Jira junto con la tarea. La estimación del esfuerzo consideró la complejidad técnica, el tiempo requerido proyectado para implementarlo y las interdependencias entre las funcionalidades, lo que permitió una planificación objetiva del desarrollo.

La planificación del proyecto se realizó mediante un diagrama de Gantt. Un diagrama de Gantt es una herramienta visual que muestra la secuencia de las tareas, sus dependencias y los plazos de implementación estimados. Este diagrama permite visualizar el cronograma del proyecto, facilitando la identificación de hitos y la gestión de recursos. En este caso, se utilizó para planificar las historias de usuario y su implementación en iteraciones o sprints, asegurando que todas las funcionalidades necesarias sean contempladas y ejecutadas de manera ordenada según sus dependencias.

En la Figura \ref{fig:jira-board}, se muestra el tablero utilizado para el seguimiento del progreso de cada historia de usuario, mientras que la Figura \ref{fig:gantt-chart} presenta el diagrama de Gantt que ilustra la secuencia de las tareas, sus dependencias y los plazos de implementación estimados. Con la planificación armada, se estimó que el desarrollo del sistema tendría una duración de 6 semanas, con un total de 28 historias de usuario a implementar. Esta estimación de tiempo corresponde exclusivamente al tiempo de codificación de funcionalidades del prototipo, ya que, posterior a la etapa de generación de código, se procede a la fase codificación y ejecución de pruebas automátizadas y validación manual del sistema, donde es posible a su vez que se deban realizar ajustes o correcciones de programación en función de los resultados obtenidos.

\begin{figure}[!htpb]
  \centering
  \includegraphics[width=0.8\textwidth]{Figures/jira-board.png}
  \caption{Tablero de Jira para la gestión de historias de usuario}
  \label{fig:jira-board}
\end{figure}

% TODO: resolve the content and format of these images. 

\begin{figure}[!htpb]
  \centering
  \includegraphics[width=0.8\textwidth]{Figures/gantt-chart.png}
  \caption{Diagrama de Gantt para la planificación del proyecto}
  \label{fig:gantt-chart}
\end{figure}

Concluidas las fases de modelado de requerimientos y planificación, se establece la base para la siguiente etapa del proceso. El conjunto de requerimientos, casos de uso y la planificación detallada con las historias de usuario constituyen la referencia que guiará las fases de diseño, implementación y pruebas del sistema. En el próximo capítulo, se abordará el diseño de la arquitectura y los componentes del sistema, donde se definirán las soluciones tecnológicas y la estructura del software que implementará los requerimientos establecidos.
