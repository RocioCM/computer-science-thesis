\chapter[Implementación]{Implementación}
\label{cp:implementation}

\parindent0pt

- la implementacion contempla la generación de código, que es la ultima etapa de bajada de la v, pero también esta fuertemente ligada a la primer etapa de pruebas de la segunda mitad, pruebas unitarias.
- Explicar que se realizó desarrollo y unit testing de forma conjunta e intercalada y por qué
- Contar que la implementación fue la etapa más larga de la V y se llevó a cabo siguiendo la planificación hecha a partir de las user stories en jira.
- Contar que seguimos el proceso con jira, siguiendo la evolución del proyecto y comparando con la planificación inicial. Contar que hubieron desvíos debido a cursado y viajes imprevistos, pero igualmente se logró el objetivo.

Seccion desarrollo tecnico:

- Al ser un sistema modular pero interconectado, el desarrollo se realizó por módulo desde adentro hacia afuera y no por dominio. Es decir, primero se programó el 100\% de los smart contracts y sus tests. Después se implementó el 100\% de la API y luego el 100\% del frontend. Justificar por qué hice esto?
- Unit tests de frontend y api y contratos se desarrollaron junto con los sistemas y se revisaron y fortalecieron (aumento del coverage) luego de finalizar el desarrollo de funcionalidades. Siempre hay idas y vueltas al realizar la integración, pero ese fue el flujo de trabajo en general. Nomás mencionar esto de los unit tests pero avisar que se va a profundizar sobre las pruebas unitarias llevadas a cabo en la sección de pruebas.
- A nivel dominio, el sistema se implementó siguiendo el ciclo de vida del vidrio: primero productor primario, luego secundario, luego consumidor, por último reciclador y recién luego de la cadena completa se hizo la funcionalidad de seguimiento end-to-end.
- Contar si hubieron dificultades en el proceso de integración.

Seccion despliegue:

- Cuando se tuvo el software ejecutable y testeado con unit testing en un entorno local, se procedió al despliegue en un entorno de pruebas. 
- Contar sobre las plataformas donde se desplegó cada módulo, el por qué (son gratuitas y ya, me sirve para un trabajo académico). Contar que no hace falta salir a produ, con un entorno de pruebas es suficiente para demostrar el funcionamiento del sistema y que se puede escalar a producción si se desea.
- Explicar que se configuraron herramientas de despliegue en docker para modularización, evitar conflictos de dependencias y que en el futuro se pueda transicionar fácilmente a un entorno productivo sin problemas ni necesidad de cambios.

Sección documentación:

- Un buen software debe estar bien documentado, resaltar su importancia como parte del proceso de desarrollo.
- Explicar que en el mismo código se encuentra documentado el proceso de setup del proyecto en un entorno local y el despliegue en producción.
- Contar que en los comentarios del proyecto se explica el funcionamiento del codigo inline. 
- Cada repo tiene su readme con requerimientos, estructura y explicaciones de proceso de despliegue, correr local y correr tests.
- Agregar que se configuró un openapi para la documentación de la api backend (interfaz backend) para que en el futuro se puedan desarrollar nuevas apps frontend que se conecten al mismo sistema interno. 

- Transicionar a la siguiente etapa, explicar que en la etapa de pruebas se va a profundizar en el proceso de pruebas unitarias, de integración y de aceptación del usuario, que fueron llevadas a cabo a partir del software implementado y que se realizaron pruebas exhaustivas para asegurar la calidad y funcionamiento correcto del sistema.
