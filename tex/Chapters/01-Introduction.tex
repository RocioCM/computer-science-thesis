\chapter[Introducción]{Introducción}
\label{cp:introduction}

\parindent0pt

Antes de comenzar con la descripción del trabajo realizado, es importante contextualizar el problema abordado y los objetivos que se pretende alcanzar. En este capítulo se introduce el contexto y la motivación del trabajo (Sección \ref{sec:motivation}), a continuación se detallan los objetivos (Sección \ref{sec:goals}) y, finalmente, se expone la estructura general del documento con el fin de guiar al lector a través del contenido desarrollado en los capítulos posteriores.

\section{Motivación}
\label{sec:motivation}

El mundo se enfrenta a un desafío ambiental sin precedentes: la gestión insostenible de los recursos naturales. La producción y consumo masivos de bienes generan un volumen creciente de residuos, lo que pone en riesgo la salud del planeta y el bienestar de las generaciones futuras \cite{IPCC2022, pelegri2021ipcc}. En este contexto, la transición hacia una economía circular se presenta como una solución prometedora para mitigar este impacto y construir un futuro más sostenible \cite{clima2022book}. Este modelo económico busca maximizar el valor de los recursos a lo largo de su ciclo de vida, minimizando el desperdicio y reintroduciendo los materiales en los sistemas de producción \cite{da2022economia, melendez2021economia}. Sin embargo, el principal desafío para lograr una economía circular radica en la falta de transparencia y trazabilidad dentro de las cadenas de suministro tradicionales.

Esta falta de visibilidad dificulta la capacidad para identificar oportunidades de reutilización y reciclaje, responsabilizar a las industrias por su impacto ambiental y empoderar a los consumidores para que tomen decisiones informadas. 

Investigaciones previas han explorado diversas tecnologías para mejorar la trazabilidad de la cadena de suministro, incluidos códigos de barras, identificación por radiofrecuencia (\gls{rfid}, \textit{Radio Frequency Identification}) y redes de sensores. Estas tecnologías ofrecen cierto nivel de capacidad de seguimiento; sin embargo, a menudo están limitadas por factores como la falta de estandarización, la fragmentación de información y la vulnerabilidad a manipulaciones \cite{schuitemaker2020product}.

En los últimos años, la tecnología \textit{blockchain} ha surgido como una solución prometedora para abordar estas limitaciones \cite{baralla2023waste, alnuaimi2023blockchain}. Sus características principales, como el registro de datos distribuido, la inmutabilidad y la transparencia, la convierten en una plataforma ideal para registrar y rastrear el movimiento de mercancías a lo largo de la cadena de suministro \cite{baralla2023waste}. Diversos estudios han explorado la aplicación de tecnología \textit{blockchain} para la trazabilidad de la cadena de suministro, demostrando su potencial para mejorar la transparencia y la responsabilidad dentro de estos sistemas. Ejemplos de estas aplicaciones incluyen la creación de un registro inmutable del origen de los productos para verificar su autenticidad y combatir la falsificación \cite{bulkowska2023implementation}, el rastreo de materiales a lo largo de la cadena de suministro para apoyar una economía circular \cite{baralla2023waste}, la optimización de la logística y la gestión de inventario mediante información en tiempo real \cite{signeblock2024} y la promoción de prácticas sostenibles al identificar productos con menor impacto ambiental \cite{bulkowska2023implementation}.

Las investigaciones realizadas hasta el momento reconocen el potencial de \textit{blockchain} para la trazabilidad de la cadena de suministro, pero muchas soluciones propuestas se enfocan únicamente en esta tecnología \cite{baralla2023waste, bulkowska2023implementation}, lo que limita su aplicabilidad en contextos donde se requiere la integración con sistemas de gestión tradicionales y tecnologías complementarias. Además, la mayoría de los estudios se centran en casos de uso específicos, como la industria alimentaria o farmacéutica, dejando una brecha significativa en la aplicación de \textit{blockchain} para mejorar la trazabilidad en otros sectores, como el reciclaje de vidrio.

En Latinoamérica, el vidrio representa el 5\% de los residuos sólidos urbanos \cite{cepal2021economia}, y solo el 20\% de este vidrio se recicla \cite{verallia2022whitebook}. La baja tasa de reciclaje de vidrio en la región se debe a la falta de infraestructura y sistemas de gestión adecuados, así como a la falta de conciencia y educación sobre la importancia del reciclaje. Mejorar la trazabilidad en la cadena de suministro del vidrio facilita su reciclaje, ayudando a promover una economía circular sostenible en la región. Al visibilizar el flujo de materiales, promover prácticas de reciclaje y facilitar la información y procesos a los usuarios, es posible reducir la generación de residuos, disminuir la extracción de materias primas vírgenes y fomentar la reutilización de materiales en la producción de nuevos envases de vidrio.

Teniendo en consideración que la actividad económica principal de la provincia de Mendoza es la producción de vino, la cadena de suministro del vidrio adquiere una relevancia particular al proveer los envases para el embotellado. En este contexto, la falta de trazabilidad en dicha cadena es una problemática local y concreta cuya solución puede tener un impacto significativo en la sostenibilidad de la industria vitivinícola y, por extensión, en la economía regional. Por lo tanto, el presente trabajo se centra específicamente en la cadena de suministro y el reciclaje de los envases de vidrio, dada la importancia de este material reciclable en la economía circular y su papel en el desarrollo de la industria vitivinícola local.

Este trabajo tiene como objetivo desarrollar una solución de trazabilidad de envases de vidrio basada en la tecnología \textit{blockchain}, con el fin de mejorar la transparencia y la sostenibilidad a lo largo de todo su ciclo de vida. La solución propuesta busca abordar las limitaciones de las tecnologías existentes, al ofrecer una plataforma que permite a los actores de la cadena rastrear y verificar el origen, el movimiento y el estado de los envases. Para lograrlo, se plantea un enfoque abierto que integra la tecnología \textit{blockchain} con el Internet de las Cosas (\gls{iot}, \textit{Internet of Things}) y los sistemas de gestión tradicionales. Esta combinación permite aprovechar los datos en tiempo real provenientes de sensores para una visión más completa y confiable del producto, mientras que su compatibilidad con las prácticas comerciales existentes facilita su adopción. En última instancia, se espera que esta implementación represente un modelo factible y práctico para mejorar la trazabilidad de la cadena de suministro, contribuyendo a la transición hacia una economía circular sostenible. De este modo, se busca proporcionar una solución concreta y aplicable en el ecosistema mendocino, que a su vez pueda servir en un futuro como modelo para adaptarse a otras industrias y a una variedad de materiales reciclables.

\section{Objetivos}
\label{sec:goals}

El objetivo general de este trabajo consiste en hacer uso de \textit{blockchain} como tecnología de vanguardia para el desarrollo de una aplicación prototipo destinada a mejorar la trazabilidad en modelos de economía circular orientados al reciclaje de vidrio. A partir de este objetivo general, se definen los siguientes objetivos específicos:

\begin{itemize}
	\item \textbf{Objetivo 1}: entender los procesos de adopción de tecnologías tales como \textit{blockchain} y las capacidades actuales en la región para el uso de sistemas de trazabilidad.
	\item \textbf{Objetivo 2}: en lo referido a las Ciencias de la Computación, se busca desarrollar una aplicación prototipo funcional basada en tecnología \textit{blockchain}. Esto permitirá la trazabilidad transparente, segura y en tiempo real de la gestión de residuos, en particular el vidrio, desde su generación hasta su disposición final, con el fin de garantizar el cumplimiento normativo, mejorar la eficiencia operativa y aumentar la confianza entre todos los actores involucrados en el proceso.
\end{itemize}

\section{Estructura general del documento}
\label{sec:document-structure}

El presente documento se encuentra organizado en capítulos, cada uno de los cuales aborda un aspecto específico del trabajo realizado. En primer lugar, en el Capítulo \ref{cp:theoretical-framework}, se introduce el marco teórico, conceptos básicos relacionados con el problema y la tecnología utilizada, para luego continuar con un análisis de las soluciones existentes y los antecedentes académicos relevantes que contextualizan el trabajo. A continuación, en el Capítulo \ref{cp:methodology} se detalla la planeación del trabajo y la metodología adoptada. En el Capítulo \ref{cp:modelling} se describe el proceso de modelado de los requerimientos de la solución propuesta. En el Capítulo \ref{cp:design} se describe el diseño del sistema y en el Capítulo \ref{cp:implementation} se detalla su implementación. Posteriormente, en el Capítulo \ref{cp:testing} se presentan las pruebas realizadas y, finalmente, en el Capítulo \ref{cp:conclusions} se presentan las reflexiones obtenidas, los desafíos encontrados y las perspectivas futuras del proyecto.

Adicionalmente, al final del documento se incluyen una serie de apéndices como lectura opcional. El Apéndice \ref{cp:annex-content} incluye contenido externo como la demostración del prototipo, el código fuente y la documentación técnica del sistema. El Apéndice \ref{cp:verallia-interview} contiene una transcripción de la entrevista realizada a una empresa referente en la industria del vidrio en Mendoza. En el Apéndice \ref{cp:europe-trip} se presenta una minuta de un viaje de investigación realizado para estudiar sistemas de reciclaje y economía circular. El Apéndice \ref{cp:user-flows} incluye los flujos de usuario detallados para cada uno de los actores del sistema. El Apéndice \ref{cp:tests-execution-results} presenta los casos de prueba detallados y resultados obtenidos de cada etapa de la verificación del sistema. Finalmente, se incluye un glosario de términos específicos y abreviaturas que pueden resultar útiles para el lector a lo largo del documento.
