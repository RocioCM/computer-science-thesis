\documentclass[main.tex]{subfiles}
\usepackage[utf8]{inputenc}
\usepackage[
backend=biber,
style=alphabetic,
sorting=ynt
]{biblatex}


\addbibresource{references.bib} %Imports bibliography file

% Document information
\title{Marco Teórico}
\author{Rocío Mena}
\date{\today}

\begin{document}

\maketitle

\section{Problema ambiental}

En la actualidad nos encontramos ante un problema ambiental sin precedente: el cambio climático. Este fenómeno, causado principalmente por la actividad humana, ha generado un aumento de las temperaturas medias en el planeta, así como una mayor recurrencia de fenómenos meteorológicos extremos. Estos efectos adversos ya están generando pérdidas y daños tanto en la naturaleza como en las sociedades humanas, exacerbando la vulnerabilidad de los sectores y regiones más expuestos \cite{clima2022book}. 

Según el consenso científico actual, el aumento de las temperaturas medias que sufre el planeta está causado, casi en su totalidad, por la acumulación de gases de efecto invernadero (GEI) en la atmósfera, como resultado de la actividad humana \cite{IPCC2022}. Esta evidencia se fundamenta en el rápido aumento de gases como el dióxido de carbono y el metano, donde aproximadamente dos tercios del dióxido de carbono provienen de la quema de combustibles fósiles, mientras que el metano tiene una contribución significativa desde la agricultura y la gestión de residuos \cite{pelegri2021ipcc}.

Las consecuencias del cambio climático en términos del aumento de las temperaturas medias y la mayor recurrencia de fenómenos meteorológicos extremos son significativas, con impactos económicos sustanciales que varían según las regiones y los grupos económicos \cite{IPCC2022}. Estos efectos adversos ya están generando pérdidas y daños tanto en la naturaleza como en las sociedades humanas, exacerbando la vulnerabilidad de los sectores y regiones más expuestos \cite{pelegri2021ipcc}.

Para mitigar y revertir los efectos del cambio climático, es esencial implementar políticas globales y locales efectivas. El IPCC (Grupo Intergubernamental de Expertos sobre el Cambio Climático) destaca la necesidad urgente de políticas regulatorias, fiscales, sociales y estructurales que incentiven una transición hacia modelos de producción y consumo más sostenibles. Entre estas medidas se incluyen la internalización de las externalidades de las emisiones de carbono a través de impuestos y sistemas de comercio de emisiones, así como la protección y apoyo a sectores vulnerables frente a los cambios estructurales \cite{IPCC2022}.

Sin embargo, el futuro del cambio climático está marcado por una alta incertidumbre. Pelegrí señala varios factores críticos que influirán en la evolución climática futura, como la capacidad de los océanos y la biosfera terrestre para absorber dióxido de carbono, que podría disminuir con el tiempo, convirtiéndose potencialmente en fuentes de emisión en lugar de sumideros. Además, la dinámica de las grandes masas de hielo antártico y la incertidumbre en el ciclo del agua, incluidos los cambios en la formación de nubes y sus efectos en el albedo y el efecto invernadero, complican aún más las proyecciones climáticas \cite{pelegri2021ipcc}.

Este panorama subraya la urgencia de una respuesta global coordinada y basada en evidencias científicas. La humanidad enfrenta el desafío de adaptarse a un clima cambiante y mitigar sus impactos negativos. En el informe publicado por el IPCC en 2022, se destaca la importancia de todo tipo de acciones, desde acciones individuales de consumo sostenible hasta políticas internacionales ambiciosas. El informe afirma que todos los niveles de la sociedad tienen un papel crucial en asegurar un futuro sostenible para las generaciones venideras, y enfatiza que es fundamental la colaboración entre los líderes políticos, la sociedad civil y el sector privado para enfrentar este desafío global de manera inclusiva y equitativa \cite{IPCC2022, pelegri2021ipcc}. En este contexto complejo que requiere replantear nuestro modelo de producción y consumo mundial para hacerlo más sostenible y resiliente, la economía circular surge como una modelo prometedor. 

\section{Economía circular}

La economía circular es un enfoque alternativo al modelo económico lineal tradicional, que se basa en la extracción de recursos naturales, la producción de bienes y servicios, y la eliminación de residuos. En contraste, la economía circular propone un sistema en el que los recursos se mantienen en uso durante el mayor tiempo posible, se reciclan y se reutilizan, minimizando la generación de residuos y reduciendo la extracción de materias primas \cite{ellenmacarthurfoundation2022}. Este enfoque se basa en los principios de diseño ecológico, la reutilización de materiales y la regeneración de los sistemas naturales, con el objetivo de crear un sistema económico más sostenible y resiliente.

Desde la Revolución Industrial, la economía global ha operado principalmente bajo un modelo de "extraer, producir y consumir", caracterizado por la explotación de recursos naturales y la generación masiva de residuos \cite{cerda2016economia}. Este enfoque, aunque ha impulsado un crecimiento económico sin precedentes, ha llevado al agotamiento de recursos críticos y a una crisis ambiental cada vez más evidente.

La economía circular propone un cambio radical al concebir los sistemas productivos y de consumo como ciclos cerrados, donde los materiales y recursos se mantienen en uso durante el mayor tiempo posible, se reciclan y se reutilizan al final de su vida útil \cite{circular2017economia}. Este enfoque no solo busca minimizar la generación de residuos, sino también maximizar el valor de los recursos a lo largo de su ciclo de vida. El eco-diseño juega un papel central en este proceso, promoviendo la creación de productos y servicios que sean más eficientes en el uso de recursos, renovables, reciclables y menos perjudiciales para el medio ambiente.

Para efectuar esta transformación hacia la economía circular, es esencial adoptar no solo innovaciones tecnológicas sino también cambios organizativos y sociales significativos \cite{espanacircular2030}. Esto implica desarrollar nuevos conocimientos y tecnologías que apoyen procesos productivos más eficientes y menos intensivos en recursos. Además, se requiere reestructurar cadenas de valor existentes y fomentar la colaboración entre sectores industriales y actores gubernamentales para promover prácticas empresariales más sostenibles.

El impacto potencial de la economía circular va más allá de la mitigación ambiental. También se espera que genere nuevas oportunidades económicas y de empleo al incentivar la innovación en productos y servicios circulares. Las empresas que adoptan principios de economía circular no solo pueden reducir costos operativos y riesgos ambientales, sino también fortalecer su posicionamiento en mercados globales cada vez más conscientes del medio ambiente.

Un elemento clave dentro de la economía circular son los principios de las "3R": Reducir, Reutilizar y Reciclar \cite{cerda2016economia}. Reducir implica minimizar el consumo de recursos y energía a través de mejoras en la eficiencia y eco-eficiencia. Reutilizar se refiere a la prolongación de la vida útil de productos y materiales mediante el segundo uso y la reparación. Reciclar, por otro lado, implica recuperar materiales valiosos de residuos para reintroducirlos en la cadena productiva.

Además de los beneficios ambientales y económicos, la economía circular también promueve un consumo más responsable y consciente, fomentando patrones de consumo que sean menos intensivos en recursos y generadores de residuos. Esto no solo contribuye a la conservación de los recursos naturales, sino que también fortalece la resiliencia de las comunidades frente a los desafíos globales emergentes.

Al desafiar el paradigma de "usar y desechar" con un enfoque en la regeneración y eficiencia de recursos, este modelo se posiciona como un catalizador clave para abordar los desafíos interconectados de la sostenibilidad ambiental y el progreso económico en el siglo XXI.

\section{Cadena de suministro y economía circular}

1. Introducción al Concepto de Economía Circular

    Definición y principios de la economía circular.
    Importancia y beneficios de implementar la economía circular.
    Contexto actual y problemas asociados con los sistemas lineales de producción y consumo.

2. Trazabilidad de Materiales

    Concepto de trazabilidad y su importancia en la gestión de materiales.
    Métodos tradicionales de trazabilidad y sus limitaciones.
    Aplicación de tecnologías emergentes para mejorar la trazabilidad.

3. Blockchain: Fundamentos y Aplicaciones

    Definición y características principales de la tecnología blockchain.
    Principios de descentralización, transparencia e inmutabilidad.
    Casos de uso relevantes de blockchain en diferentes industrias.

4. Integración de Blockchain en la Trazabilidad de Materiales

    Ventajas de utilizar blockchain para la trazabilidad de materiales.
    Casos de estudio y proyectos piloto que demuestran la efectividad de blockchain en la trazabilidad.
    Desafíos técnicos, económicos y regulatorios de implementar blockchain en sistemas de trazabilidad.

5. Economía Circular y Tecnología Blockchain

    Sinergias entre la economía circular y la tecnología blockchain.
    Cómo blockchain puede facilitar la transición hacia modelos circulares de negocio.
    Ejemplos específicos de cómo la trazabilidad mediante blockchain puede cerrar ciclos de materiales.

6. Marco Conceptual y Modelos Teóricos

    Desarrollo de un marco conceptual para la aplicación de blockchain en la trazabilidad de materiales.
    Revisión crítica de modelos teóricos relevantes para tu investigación.
    Adaptación de modelos existentes para el contexto de la economía circular y blockchain.

7. Implicaciones Sociales, Ambientales y Económicas

    Impacto potencial de la implementación de blockchain en la trazabilidad de materiales sobre aspectos sociales, ambientales y económicos.
    Consideraciones éticas y de responsabilidad en la gestión de datos en blockchain.
    Perspectivas futuras y recomendaciones para la implementación práctica.

\section{Cadena de suministros y economía circular}

\section{Regulaciones sobre sustentabilidad y trazabilidad}

\section{Trazabilidad en la cadena de suministro}

\section{Integración de tecnologías para la trazabilidad}

\section{Tecnología blockchain para la trazabilidad}

\section{Aplicaciones de blockchain en la cadena de suministro}

\section{Tecnología blockchain para la sustentabilidad}

\end{document}
