\documentclass[main.tex]{subfiles}
\usepackage[utf8]{inputenc}
\usepackage[
backend=biber,
style=alphabetic,
sorting=ynt
]{biblatex}
\usepackage{subfiles}

\addbibresource{../references.bib} %Imports bibliography file

% Document information
\title{Marco Teórico}
\author{Rocío Mena}
\date{\today}

\begin{document}

\maketitle

\section{Problema ambiental}

En la actualidad nos encontramos ante un problema ambiental sin precedente: el cambio climático. Este fenómeno, causado principalmente por la actividad humana, ha generado un aumento de las temperaturas medias en el planeta, así como una mayor recurrencia de fenómenos meteorológicos extremos. Estos efectos adversos ya están generando pérdidas y daños tanto en la naturaleza como en las sociedades humanas, exacerbando la vulnerabilidad de los sectores y regiones más expuestos \cite{clima2022book}. 

Según el consenso científico actual, el aumento de las temperaturas medias que sufre el planeta está causado, casi en su totalidad, por la acumulación de gases de efecto invernadero (GEI) en la atmósfera, como resultado de la actividad humana \cite{IPCC2022}. Esta evidencia se fundamenta en el rápido aumento de gases como el dióxido de carbono y el metano, donde aproximadamente dos tercios del dióxido de carbono provienen de la quema de combustibles fósiles, mientras que el metano tiene una contribución significativa desde la agricultura y la gestión de residuos \cite{pelegri2021ipcc}.

Las consecuencias del cambio climático en términos del aumento de las temperaturas medias y la mayor recurrencia de fenómenos meteorológicos extremos son significativas, con impactos económicos sustanciales que varían según las regiones y los grupos económicos \cite{IPCC2022}. Estos efectos adversos ya están generando pérdidas y daños tanto en la naturaleza como en las sociedades humanas, exacerbando la vulnerabilidad de los sectores y regiones más expuestos \cite{pelegri2021ipcc}.

Para mitigar y revertir los efectos del cambio climático, es esencial implementar políticas globales y locales efectivas. El IPCC (Grupo Intergubernamental de Expertos sobre el Cambio Climático) destaca la necesidad urgente de políticas regulatorias, fiscales, sociales y estructurales que incentiven una transición hacia modelos de producción y consumo más sostenibles. Entre estas medidas se incluyen la internalización de las externalidades de las emisiones de carbono a través de impuestos y sistemas de comercio de emisiones, así como la protección y apoyo a sectores vulnerables frente a los cambios estructurales \cite{IPCC2022}.

Sin embargo, el futuro del cambio climático está marcado por una alta incertidumbre. Pelegrí señala varios factores críticos que influirán en la evolución climática futura, como la capacidad de los océanos y la biosfera terrestre para absorber dióxido de carbono, que podría disminuir con el tiempo, convirtiéndose potencialmente en fuentes de emisión en lugar de sumideros. Además, la dinámica de las grandes masas de hielo antártico y la incertidumbre en el ciclo del agua, incluidos los cambios en la formación de nubes y sus efectos en el albedo y el efecto invernadero, complican aún más las proyecciones climáticas \cite{pelegri2021ipcc}.

Este panorama subraya la urgencia de una respuesta global coordinada y basada en evidencias científicas. La humanidad enfrenta el desafío de adaptarse a un clima cambiante y mitigar sus impactos negativos. En el informe publicado por el IPCC en 2022, se destaca la importancia de todo tipo de acciones, desde acciones individuales de consumo sostenible hasta políticas internacionales ambiciosas. El informe afirma que todos los niveles de la sociedad tienen un papel crucial en asegurar un futuro sostenible para las generaciones venideras, y enfatiza que es fundamental la colaboración entre los líderes políticos, la sociedad civil y el sector privado para enfrentar este desafío global de manera inclusiva y equitativa \cite{IPCC2022, pelegri2021ipcc}. En este contexto complejo que requiere replantear nuestro modelo de producción y consumo mundial para hacerlo más sostenible y resiliente, la economía circular surge como una modelo prometedor. 

\section{Economía lineal}

La economía lineal es el modelo económico predominante en la actualidad, caracterizado por la extracción de recursos naturales, la producción de bienes y servicios, y la eliminación de residuos. Este enfoque, basado en el paradigma de "usar y desechar", ha sido fundamental para el crecimiento económico y la prosperidad material de las sociedades modernas. Sin embargo, este modelo lineal presenta una serie de limitaciones y externalidades negativas que plantean desafíos significativos para la sostenibilidad ambiental y económica a largo plazo \cite{cerda2016economia}.

Uno de los principales problemas asociados con la economía lineal es la generación masiva de residuos y la contaminación ambiental. La producción y el consumo de bienes y servicios generan una gran cantidad de desechos, muchos de los cuales son difíciles de eliminar de manera segura y sostenible. La acumulación de residuos plásticos, químicos y electrónicos en el medio ambiente representa una amenaza significativa para la salud humana y el equilibrio de los ecosistemas naturales \cite{clima2022book}.

Además, la economía lineal se basa en la extracción y consumo de recursos naturales finitos, lo que conduce a la sobreexplotación de los ecosistemas y la degradación de los recursos naturales. La deforestación, la pérdida de biodiversidad, la contaminación del agua y la escasez de recursos críticos son algunas de las consecuencias negativas de este enfoque extractivo. A medida que la población mundial y la demanda de recursos continúan creciendo, la insostenibilidad de este modelo se vuelve cada vez más evidente \cite{clima2022book}.

Otro aspecto crítico de la economía lineal es su dependencia de los combustibles fósiles y la energía no renovable. La quema de carbón, petróleo y gas natural para la producción de energía es una de las principales fuentes de emisiones de gases de efecto invernadero, contribuyendo significativamente al cambio climático y sus impactos asociados. La transición hacia fuentes de energía renovable y procesos de producción más eficientes es fundamental para reducir la huella ambiental de la economía lineal y mitigar los efectos del cambio climático \cite{clima2022book, onu2024ods}.

\section{Economía circular}

La economía circular es un enfoque alternativo al modelo económico lineal tradicional, que se basa en la extracción de recursos naturales, la producción de bienes y servicios, y la eliminación de residuos. En contraste, la economía circular propone un sistema en el que los recursos se mantienen en uso durante el mayor tiempo posible, se reciclan y se reutilizan, minimizando la generación de residuos y reduciendo la extracción de materias primas \cite{ellenmacarthurfoundation2022}. Este enfoque se basa en los principios de diseño ecológico, la reutilización de materiales y la regeneración de los sistemas naturales, con el objetivo de crear un sistema económico más sostenible y resiliente.

Desde la Revolución Industrial, la economía global ha operado principalmente bajo un modelo lineal de "extraer, producir y consumir", caracterizado por la explotación de recursos naturales y la generación masiva de residuos \cite{cerda2016economia}. Este enfoque, aunque ha impulsado un crecimiento económico sin precedentes, ha generado sobreexplotación y degradación de los ecosistemas. La deforestación, pérdida de biodiversidad, contaminación del agua y escasez de recursos no renovables son algunas de las consecuencias negativas de este enfoque extractivo a gran escala. A medida que la población mundial y la demanda de recursos continúan creciendo, la insostenibilidad de este modelo a largo plazo se hace cada vez más evidente \cite{clima2022book}.

La economía circular propone un cambio radical de paradigma, al concebir los sistemas productivos y de consumo como ciclos cerrados, donde los materiales y recursos se mantienen en uso durante el mayor tiempo posible, se reciclan y se reutilizan al final de su vida útil \cite{circular2017economia}. Este enfoque busca minimizar la generación de residuos y también maximizar el valor de los recursos a lo largo de su ciclo de vida.

El impacto potencial de la economía circular va más allá de la mitigación ambiental. También suele generar nuevas oportunidades económicas y de empleo al incentivar la innovación en productos y servicios circulares. El eco-diseño juega un papel importante en este modelo, promoviendo la creación de productos y servicios más eficientes en el uso de recursos, más fáciles de reciclar y menos perjudiciales para el medio ambiente. Las empresas que adoptan principios de economía circular pueden reducir riesgos ambientales, mientras que fortalecen su posicionamiento en mercados globales cada vez más conscientes del medio ambiente.

El modelo circular también promueve una cultura de consumo más responsable y consciente, fomentando patrones de consumo que valoran la calidad y la sostenibilidad de los productos. Para los consumidores, los principios de la economía circular se pueden resumir en el lema de las "3R": Reducir, Reutilizar y Reciclar \cite{cerda2016economia}. Reducir implica minimizar el consumo de recursos y energía en la producción y el uso de productos. Reutilizar se refiere a prolongar la vida útil de los productos mediante su segundo uso o reparación. Reciclar implica recuperar materiales valiosos de los residuos para reintroducirlos en la cadena productiva, reduciendo la necesidad de extraer nuevos recursos.

Para efectuar la transición de economía lineal hacia economía circular, es necesario adoptar innovaciones tecnológicas y también cambios organizativos y sociales significativos \cite{espanacircular2030}. Esto implica desarrollar nuevos conocimientos y tecnologías que apoyen procesos productivos más eficientes y menos intensivos en recursos. Además, se requiere reestructurar cadenas de valor existentes y fomentar la colaboración entre sectores industriales y actores gubernamentales para promover prácticas empresariales más sostenibles.







\section{Cadena de suministro}

La cadena de suministro es un concepto fundamental en la economía global, que se refiere al conjunto de actividades y procesos involucrados en la producción, distribución y venta de bienes y servicios. Desde la extracción de materias primas hasta la entrega de productos terminados a los consumidores, la cadena de suministro abarca una amplia gama de actividades y actores, incluidos proveedores, fabricantes, distribuidores, minoristas y consumidores \cite{christopher2016logistics}.

La cadena de suministro es un sistema complejo y dinámico que se extiende a lo largo de múltiples etapas y ubicaciones geográficas, con interacciones entre diferentes actores y procesos. La eficiencia y la efectividad de la cadena de suministro son fundamentales para el éxito de las organizaciones en un entorno empresarial cada vez más competitivo y globalizado \cite{christopher2016logistics}.

La gestión de la cadena de suministro implica la coordinación y optimización de los flujos de materiales, información y dinero a lo largo de toda la cadena, con el objetivo de satisfacer las necesidades de los clientes de manera eficiente y rentable. Esto incluye la planificación de la demanda, la gestión de inventarios, la programación de la producción, la logística de distribución y la gestión de relaciones con los proveedores y los clientes \cite{christopher2016logistics}.

La cadena de suministro es un componente crítico de la economía global, ya que conecta a los productores con los consumidores y facilita el intercambio de bienes y servicios en todo el mundo. La eficiencia y la resiliencia de la cadena de suministro son fundamentales para garantizar la disponibilidad y la calidad de los productos en el mercado, así como para minimizar los costos y los impactos ambientales asociados con la producción y distribución de bienes \cite{christopher2016logistics}.

En el contexto de la economía circular, la cadena de suministro juega un papel crucial en la implementación de prácticas sostenibles y circulares. La adopción de principios de economía circular en la cadena de suministro puede ayudar a reducir la generación de residuos, minimizar la extracción de recursos naturales y promover la reutilización y el reciclaje de materiales valiosos \cite{melendez2021economia}.

\section{Producción y reciclaje de Vidrio}

El vidrio es un material ampliamente utilizado en la industria del envasado de alimentos y bebidas debido a sus propiedades únicas como la transparencia, durabilidad y reciclabilidad. La producción de vidrio implica la fusión de materias primas como arena, sosa y caliza a altas temperaturas para formar un material sólido y homogéneo. Este material se moldea luego en envases de vidrio mediante técnicas de soplado o prensado \cite{verallia2022vidrio}.

Este material es uno de los pocos que puede reciclarse infinitamente sin perder calidad. Al reciclar vidrio, se puede reutilizar el 100\% del material reciclado para producir nuevos envases sin necesidad de añadir materias primas vírgenes. Esto no solo reduce la generación de residuos y la extracción de recursos naturales, sino que también disminuye la huella de carbono asociada con la producción de nuevo vidrio \cite{verallia2022vidrio}.

La cadena de suministro del vidrio abarca múltiples etapas, desde la extracción de materias primas hasta la producción, distribución y reciclaje de envases de vidrio. La trazabilidad en esta cadena es fundamental para garantizar la calidad y sostenibilidad de los productos, así como para promover prácticas de producción y consumo responsables \cite{verallia2022vidrio}.

\subsection{Producción de Vidrio}

Verallia, una empresa internacional especializada en la producción de envases de vidrio para alimentos y bebidas, explica el proceso de producción de envases de vidrio separado en las siguientes etapas \cite{prodvidrio2024verallia}:

\begin{enumerate}
    \item \textbf{Selección de Materias Primas}: Se seleccionan y almacenan arena de sílice, sosa (carbonato de sodio) y caliza (carbonato de calcio).
    \item \textbf{Dosificación y Mezcla}: Las materias primas se pesan y mezclan en proporciones específicas para formar el lote de vidrio. En esta etapa se puede incluir una proporción de vidrio reciclado molido (calcín) a la mezcla.
    \item \textbf{Fusión}: La mezcla se funde en un horno a aproximadamente 1550°C durante aproximadamente 24 horas para obtener un líquido homogéneo.
    \item \textbf{Formación}: El vidrio fundido se corta en gotas y se moldea en las formas deseadas utilizando técnicas como soplado-soplado o prensa-soplado dependiendo del tipo de envase que se está produciendo.
    \item \textbf{Recocido}: Los envases de vidrio se enfrían lentamente para eliminar tensiones internas y asegurar su resistencia.
    \item \textbf{Inspección}: Cada envase se inspecciona para detectar defectos y asegurar que cumpla con los estándares de calidad.
    \item \textbf{Almacenamiento y Distribución}: Los envases aprobados se empaquetan y almacenan para su distribución al cliente final.
\end{enumerate}

\subsection{Reciclaje de Vidrio}

El vidrio reciclado se utiliza para producir nuevos envases, ahorrando materias primas y energía. El proceso de reciclaje incluye:

\begin{enumerate}
    \item \textbf{Recolección}: Se recogen envases de vidrio usados mediante programas de recogida selectiva.
    \item \textbf{Clasificación}: El vidrio se clasifica por color y se separa de otros materiales.
    \item \textbf{Limpieza}: El vidrio se lava y tritura en pequeños fragmentos.
    \item \textbf{Fundición}: Los fragmentos se funden para formar vidrio reciclado.
    \item \textbf{Formación}: Se moldea el vidrio fundido en nuevas formas de envases.
\end{enumerate}

\subsection{Producción y reciclaje de Vidrio en Mendoza}

En Mendoza, la industria del vidrio se centra principalmente en la producción de envases para la industria del vino y otras bebidas, así como frascos para alimentos. En la región se encuentra una única empresa dedicada a la producción y reciclaje de vidrio: Verallia. Esta empresa cubre el 100\% de la demanda de botellas y frascos de vidrio de la provincia.

Verallia es una empresa internacional líder en la producción de envases de vidrio para la industria de alimentos y bebidas. La empresa opera en más de 30 países y cuenta con una amplia gama de productos y servicios para satisfacer las necesidades de sus clientes en todo el mundo. En Mendoza, Verallia produce botellas y frascos de vidrio para una variedad de sectores, incluidos el vino, la cerveza, los licores, los alimentos y las bebidas no alcohólicas. La empresa adopta prácticas sostenibles y tecnologías innovadoras para minimizar su impacto ambiental y promover el reciclaje de vidrio \cite{verallia2022vidrio}.

A través del programa "Vidrio, una acción transparente" \cite{vidriotransparente2024mendoza}, Verallia y el Gobierno de Mendoza fomentan la recolección y reciclaje de vidrio, apoyando a la vez a organizaciones benéficas locales con los ingresos del vidrio reciclado. Este enfoque no solo beneficia al medio ambiente sino que también contribuye a la economía y el bienestar social de la región.

\section{Trazabilidad en la Cadena de Suministro}

La trazabilidad en la cadena de suministro es un aspecto clave para garantizar la transparencia y la integridad de los productos a lo largo de toda la cadena. La trazabilidad permite a las organizaciones rastrear el origen y el destino de los productos, identificar posibles problemas y tomar medidas correctivas de manera oportuna \cite{cepeda2010trazabilidad}. En el contexto de la economía circular, la trazabilidad en la cadena de suministro es fundamental para garantizar la calidad y la autenticidad de los materiales reciclados y reutilizados, así como para promover la trazabilidad de los productos a lo largo de su ciclo de vida \cite{melendez2021economia}.

La trazabilidad en la cadena de suministro se basa en la recopilación y el intercambio de información relevante sobre los productos y los procesos involucrados en su producción y distribución. Esta información puede incluir datos sobre la procedencia de los materiales, las condiciones de producción, el transporte y el almacenamiento, así como la gestión de residuos y la disposición final de los productos. La trazabilidad en la cadena de suministro puede facilitar la identificación de problemas de calidad, la gestión de riesgos y la toma de decisiones informadas en toda la cadena.

La trazabilidad es fundamental en la cadena de suministro por varias razones:

\begin{itemize}
    \item \textbf{Seguridad del consumidor:} Permite la retirada rápida de productos defectuosos o peligrosos del mercado.
    \item \textbf{Cumplimiento regulatorio:} Ayuda a cumplir con las regulaciones gubernamentales y las normativas de la industria, especialmente en sectores como alimentos, medicamentos y productos químicos.
    \item \textbf{Gestión de la calidad:} Facilita la identificación de problemas en la cadena de suministro y permite mejoras en los procesos.
    \item \textbf{Sostenibilidad:} Apoya prácticas de negocio sostenibles al asegurar que los materiales y procesos cumplen con criterios éticos y ambientales.
\end{itemize}

Implementar la trazabilidad efectiva requiere una combinación de tecnología, estándares y cooperación entre todos los actores de la cadena. Los sistemas de información avanzados, como el código de barras, RFID (identificación por radiofrecuencia) y la tecnología blockchain, son herramientas que se usan actualmente para registrar y acceder a la información del producto en tiempo real \cite{cepeda2010trazabilidad, BISWAS2023128}.

\subsection{Desafíos de la Trazabilidad}
A pesar de sus beneficios, la implementación de la trazabilidad enfrenta varios desafíos:
\begin{itemize}
    \item \textbf{Complejidad técnica:} Integrar sistemas de trazabilidad en cadenas de suministro complejas y geográficamente dispersas puede ser tecnológicamente desafiante.
    \item \textbf{Costos:} La actualización o instalación de nuevos sistemas de trazabilidad puede ser costosa para las empresas, especialmente para las pequeñas y medianas empresas (PYMES).
    \item \textbf{Cooperación entre actores:} Requiere que todos los participantes de la cadena compartan información abiertamente, lo que puede ser impedido por intereses competitivos o falta de confianza.
    \item \textbf{Privacidad y seguridad de datos:} Manejar la gran cantidad de datos generados y asegurar su protección es crucial para evitar problemas de privacidad y seguridad.
\end{itemize}

Varias industrias han implementado con éxito sistemas de trazabilidad que han resultado en mejoras significativas en eficiencia, seguridad del consumidor y sostenibilidad. Por ejemplo, en la industria alimentaria, la trazabilidad ha permitido identificar rápidamente y reducir el alcance de los retiros de productos contaminados, protegiendo así la salud pública y la imagen de las marcas involucradas.

Un enfoque emergente para abordar los desafíos de la trazabilidad en la cadena de suministro es la tecnología blockchain. La tecnología blockchain ofrece una forma segura y descentralizada de registrar y compartir información en tiempo real, lo que puede mejorar la transparencia, la integridad y la eficiencia de los sistemas de trazabilidad en la cadena de suministro.

\subfile{technologies}

\end{document}
