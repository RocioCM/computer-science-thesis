\documentclass[main.tex]{subfiles}
\usepackage[utf8]{inputenc}
\usepackage[
backend=biber,
style=alphabetic,
sorting=ynt
]{biblatex}


\addbibresource{references.bib} %Imports bibliography file

% Document information
\title{Estado del arte}
\author{Rocío Mena}
\date{\today}

\begin{document}

\maketitle

\section{Políticas orientadas a la sustentabilidad}

La transición hacia prácticas más sostenibles y respetuosas con el medio ambiente es uno de los desafíos más significativos y urgentes que enfrentan las sociedades contemporáneas. En este contexto, las políticas públicas orientadas a la sustentabilidad ecológica desempeñan un papel crucial en la modelación de estrategias que no solo buscan mitigar los efectos del cambio climático, sino también transformar la economía global hacia modelos más circulares y regenerativos. Estas políticas están diseñadas para integrar tres dimensiones del desarrollo sostenible: la económica, la social y la ambiental, ofreciendo así una hoja de ruta integral que articula la acción colectiva en torno a objetivos comunes que son tanto ambientales como socioeconómicos \cite{gil2018objetivos}.

En el ámbito de América Latina, por ejemplo, se han implementado modelos de economía circular que intentan reformular el uso y gestión de los recursos, promoviendo la reducción, reutilización y reciclaje de materiales \cite{rodriguez2023modelamiento, cepal2021economia}. Estas iniciativas son fundamentales para enfrentar los retos locales y globales de la gestión de residuos y la preservación de recursos naturales. En este sentido, la Estrategia Nacional de Consumo y Producción Sostenibles de Argentina representa un esfuerzo significativo por parte del gobierno para alinear las prácticas de consumo y producción con los Objetivos de Desarrollo Sostenible (ODS) de las Naciones Unidas, enfocándose específicamente en el ODS 12, que promueve un consumo y producción responsables \cite{sostenible2021argentina}.

Sin embargo, a pesar de los avances en la formulación de políticas, la implementación efectiva enfrenta desafíos significativos. La complejidad de las estructuras económicas existentes, la falta de compromiso a largo plazo y las necesidades de inversiones considerables para tecnologías sostenibles son solo algunas de las barreras que estos países deben superar. Además, la eficacia de estas políticas frecuentemente se ve limitada por la falta de coherencia y coordinación entre diferentes niveles de gobierno y sectores de la sociedad, así como por la insuficiencia de indicadores claros y medibles para evaluar el progreso hacia los objetivos propuestos \cite{gil2018objetivos}.

A pesar de los esfuerzos legislativos y las iniciativas implementadas, es evidente que el impacto real de las políticas públicas orientadas a la sustentabilidad ecológica, aunque positivo en cierta medida, no ha alcanzado el nivel de efectividad esperado \cite{gil2018objetivos, clima2022book}. De manera generalizada, el impacto positivo de estas políticas ha sido insuficiente para contrarrestar de manera significativa el impacto ambiental negativo de las sociedades en las que se han implementado, dentro del período acordado. Esto resalta una brecha crítica entre los objetivos propuestos y los resultados tangibles obtenidos, subrayando la necesidad urgente de reevaluar y fortalecer los mecanismos de acción y cumplimiento. La insuficiencia de estas políticas para neutralizar los efectos adversos en el tiempo establecido indica un desafío persistente en la política de desarrollo sostenible global.

En conclusión, mientras que la formulación de políticas públicas orientadas a la sustentabilidad ecológica es un paso vital hacia un desarrollo más sostenible, la clave para su éxito radica en la capacidad de implementar estos marcos de manera efectiva, garantizando que las transiciones hacia modelos circulares y sostenibles sean viables, inclusivas y beneficiosas para cada actor del modelo. A continuación se exploran una serie de políticas específicas implementadas en diversos contextos, evaluando su impacto, efectividad y las lecciones aprendidas en el proceso.

\subsection{Evolución de las Políticas de Cambio Climático y la Neutralidad Climática en la Unión Europea}

En el marco del Pacto Verde Europeo, la Unión Europea ha otorgado rango legal a su objetivo de neutralidad climática mediante la Ley Europea del Clima, impulsando una serie de políticas innovadoras para su implementación, como el paquete de medidas conocido como Objetivo 55 \cite{dormido2022cambio}.

\subsubsection{De Río a París: Hitos en la Acción Climática}
La trayectoria internacional de la acción frente al cambio climático ha evolucionado significativamente desde los acuerdos iniciales en Río de Janeiro en 1992. Este cambio refleja una mayor comprensión de la influencia humana sobre el clima y sus implicaciones económicas, culminando en la adopción de la Convención Marco de las Naciones Unidas sobre el Cambio Climático (CMNUCC). Los esfuerzos continuaron con el Protocolo de Kyoto en 1997, que estableció por primera vez compromisos de reducción de emisiones de CO2 para los países desarrollados \cite{dormido2022cambio}.

\subsubsection{El Acuerdo de París y su Implementación}
El Acuerdo de París, firmado en 2015, marca un hito histórico al ser el primer tratado internacional universal sobre el cambio climático. Este acuerdo compromete a sus signatarios a mantener el aumento de la temperatura global por debajo de los 2ºC con respecto a los niveles preindustriales y esforzarse por limitar este aumento a 1,5ºC. Además, se establecen mecanismos para la mitigación, adaptación y resiliencia ante el cambio climático, incluyendo un marco financiero y técnico para apoyar a los países más vulnerables \cite{dormido2022cambio}.

\subsubsection{Acciones del G-20 y la Economía Circular}
En respuesta a la crisis del COVID-19, los asuntos climáticos han ganado protagonismo en la agenda del G-20, reflejando la urgencia de estas políticas. Los comunicados del G-20 han resaltado la importancia de asignar recursos financieros adecuados para la mitigación del cambio climático y la adopción de tecnologías limpias para alcanzar emisiones cero, así como el acceso a fuentes de energía limpia. Además, se destaca el acuerdo para la reducción de emisiones de metano, consideradas clave en la estrategia de mitigación rápida y económica del cambio climático \cite{dormido2022cambio}.

\subsubsection{Impuestos al Carbono y Políticas Fiscales Sostenibles}
Las recomendaciones del Fondo Monetario Internacional enfatizan la necesidad de un impuesto global al carbono como la medida más eficiente para la descarbonización. A nivel nacional, se sugieren reformas en los subsidios a los combustibles fósiles y el fomento de inversiones en energías verdes, complementadas con transferencias para compensar a los hogares vulnerables por los efectos de la transición energética \cite{dormido2022cambio}.

\subsubsection{El Pacto Verde Europeo}
Adoptado en 2019, el Pacto Verde Europeo es el compromiso programático más reciente de la UE en la lucha contra el cambio climático. Este pacto busca transformar los retos climáticos y ambientales en oportunidades para todas las áreas de actuación, promoviendo una transición justa e integradora hacia una economía sostenible y climáticamente neutra para 2050. El principal desafío de implementar el Pacto Verde y otras políticas similares radica en la necesidad de una transición justa que compense a las comunidades afectadas por los cambios industriales \cite{dormido2022cambio}.

\subsection{Objetivos de Desarrollo Sostenible}
Los Objetivos de Desarrollo Sostenible (ODS), establecidos por la Organización de las Naciones Unidas (ONU) en 2015 como parte de la Agenda 2030 para el Desarrollo Sostenible, son un marco global compuesto por 17 objetivos interconectados. Estos objetivos representan un esfuerzo global para erradicar la pobreza, proteger el medio ambiente, y garantizar que todas las personas disfruten de paz y prosperidad para 2030. Son el consenso más amplio a lo que la humanidad aspira en cuanto a un futuro deseable y abordan las dimensiones económica, social y ambiental del desarrollo sostenible a través de metas concretas y cuantificables. Los objetivos se materializan en 169 metas concretas medibles a través de 230 indicadores, diseñados para ser aplicados universalmente a todos los países  \cite{onu2024ods gil2018objetivos}.

Varios ODS están directamente relacionados con la economía circular y la sostenibilidad ecológica, destacando la importancia de:

\begin{itemize}
		\item \textbf{ODS 7: Energía asequible y no contaminante.} Garantiza el acceso universal a servicios energéticos asequibles, confiables, sostenibles y modernos.
    \item \textbf{ODS 11: Ciudades y comunidades sostenibles.} Se enfoca en hacer que las ciudades y los asentamientos humanos sean inclusivos, seguros, resilientes y sostenibles.
    \item \textbf{ODS 12: Producción y Consumo Responsables} — Fomenta prácticas sostenibles que incluyen la reducción del desperdicio mediante la prevención, reciclaje y reutilización.
    \item \textbf{ODS 13: Acción por el Clima} — Insta a la adopción de medidas urgentes para combatir el cambio climático y sus efectos.
    \item \textbf{ODS 14 y 15: Vida Submarina y Terrestre} — Promueven la conservación y uso sostenible de los ecosistemas oceánicos y terrestres, esenciales para la biodiversidad y la sostenibilidad ambiental.
\end{itemize}

Los ODS ofrecen una visión integradora y global del desarrollo sostenible. Sin embargo, para que esta visión se materialice en acciones efectivas y resultados tangibles, se requiere un compromiso profundo y coordinado a nivel internacional, junto con estrategias adaptadas a las realidades locales de cada país \cite{gil2018objetivos}.
Los ODS, como un marco global para el desarrollo sostenible, necesitan acciones concertadas y compromisos políticos claros para ser efectivos. La comunidad internacional debe priorizar estos objetivos dentro de sus políticas nacionales y cooperar a nivel internacional para asegurar que los esfuerzos sean inclusivos y efectivos en todos los sectores de la sociedad \cite{onu2024ods}.

A pesar de la amplitud y la ambición de los ODS, han surgido críticas significativas respecto a su estructura y efectividad. La complejidad de su arquitectura y las limitaciones técnicas han sido puntos de preocupación, así como la falta de compromisos específicos y medibles que aseguren su cumplimiento. Los ODS han sido criticados por ser en gran parte retóricos y ambiciosos sin suficientes directrices claras para su implementación, lo que ha generado disparidades en la aplicación entre diferentes países \cite{gil2018objetivos}.

Los ODS requieren un cambio significativo en las políticas y prácticas globales, lo cual ha sido un proceso complejo y desigual entre los países. Los estados han encontrado dificultades para avanzar en la implementación de estas metas, debido en parte a la falta de indicaciones claras sobre cómo llevar a cabo estas transformaciones. Además, la voluntariedad de los ODS permite que cada país avance a su propio ritmo, lo que podría resultar en esfuerzos dispersos y no coordinados \cite{gil2018objetivos}.

A pesar de la ambición de los ODS sobre el progreso ecológico y la sostenibilidad global, el impacto real hasta la fecha ha sido mixto en comparación con las proyecciones iniciales. Según el análisis presentado en \textit{Clima} \cite{clima2022book}, la implementación de los ODS ha enfrentado desafíos significativos, incluyendo la falta de recursos financieros, políticas incoherentes a nivel nacional, y la necesidad de mayor cooperación internacional. Aunque se han hecho avances en algunos sectores, el progreso global hacia metas críticas como la reducción de emisiones de gases de efecto invernadero y la conservación de la biodiversidad ha sido insuficiente para cumplir con los objetivos establecidos para 2030. Este desfase subraya la necesidad urgente de revisar las estrategias y compromisos, tanto a niveles nacionales como internacionales, para asegurar que los avances hacia la sostenibilidad ecológica no solo sean aspiracionales sino efectivos y tangibles.

\subsubsection{Estrategias de Desarrollo Productivo Verde}
El capítulo sobre políticas de desarrollo del FUNDAR subraya la importancia de integrar dimensiones económicas, sociales y ambientales en el desarrollo. Busca avanzar hacia una estructura productiva que maximice la eficiencia en el uso de recursos y minimice el impacto ambiental. 

Las políticas sostenibles enfrentan limitaciones significativas debido a la complejidad de los sistemas económicos y la falta de un marco regulatorio coherente. La implementación efectiva de estas políticas se ve obstaculizada por estos factores, junto con la resistencia de sectores arraigados y la insuficiencia de inversiones en tecnologías limpias.

\subsection{Políticas Sustentables y Gestión de Residuos en América Latina}

\subsubsection{Introducción a la Economía Circular y su Impacto Macroeconómico}
La transición hacia una economía circular en América Latina no solo implica un cambio en la gestión de residuos sino que también ofrece una oportunidad para mejorar la macroeconomía de la región. En el estudio \textit{Modelamiento de los efectos macroeconómicos de la transición a la economía circular en América Latina} \cite{rodriguez2023modelamiento}, se exploran los impactos potenciales en el empleo, la huella climática, y el Producto Interno Bruto (PIB) mediante un modelo que simula diferentes escenarios de transición. Este modelo destacó que, con una reducción conservadora en el uso de materiales como el plástico y el cemento, los países podrían ver incrementos en el PIB de hasta 2.2\% y mejoras en el empleo de hasta 2.1\% para 2030, además de reducciones en las emisiones de gases de efecto invernadero (GEI), mostrando un potencial para recuperaciones verdes \cite{rodriguez2023modelamiento}.

\subsection{Políticas Públicas y Avances en la Gestión de Residuos}
Según el documento \textit{Economía Circular en América Latina y el Caribe: oportunidad para una recuperación transformadora} \cite{cepal2021economia}, la región ha implementado diversas políticas para fomentar la economía circular, que incluyen legislaciones sobre la responsabilidad extendida del productor y prohibiciones de plásticos de un solo uso. Estas políticas buscan no solo reducir la generación de residuos sino también promover la reutilización y el reciclado, integrando aspectos económicos, sociales y ambientales en línea con la Agenda 2030 para el Desarrollo Sostenible \cite{cepal2021economia}.

A pesar de los avances, existen desafíos significativos debido al déficit en infraestructura para la gestión de residuos y las bajas tasas de reciclaje, que están muy centradas en pocos productos, como el papel y el cartón. Estas limitaciones ofrecen, sin embargo, una ventana de oportunidad para mejorar la gestión de residuos y promover la economía circular a nivel local, utilizando cadenas productivas que potencien el desarrollo sostenible \cite{cepal2021economia}.

\subsubsection{Estrategias Legislativas y Normativas}
La implementación de leyes de responsabilidad extendida del productor en la región ha sido un paso importante para asegurar que los fabricantes asuman una parte del costo y la gestión de sus productos al final de su vida útil. Sin embargo, el impacto de estas políticas aún está limitado por la falta de un sistema integral que involucre a todos los actores, desde consumidores hasta gestores de residuos y el Estado, para garantizar una gestión eficiente de los residuos \cite{cepal2021economia}.

Según Rodríguez \cite{rodriguez2023modelamiento}, es crucial usar la experiencia adquirida en la modelación inicial para desarrollar modelos más complejos que integren los efectos ambientales. Esto incluye la adaptación del modelo a diferentes contextos macroeconómicos y sectores, lo que permitirá una aplicación más amplia de la economía circular en América Latina. Además, es esencial fortalecer la construcción de indicadores y metas físicas para una transición efectiva hacia la economía circular.

Ambos informes conciden en que la transición hacia la economía circular en América Latina presenta una oportunidad única para alinear los objetivos económicos con los ambientales y sociales, promoviendo un desarrollo sostenible y resiliente. Sin embargo, se requiere una colaboración más estrecha entre los gobiernos, la industria y la sociedad civil para superar los desafíos actuales y maximizar los beneficios de las políticas implementadas. La continuidad en la evaluación y ajuste de las políticas será crucial para garantizar que la región pueda alcanzar los objetivos establecidos para 2030 y más allá \cite{cepal2021economia}.

\subsection{Políticas Sustentables en Argentina}
En Argentina, las políticas de desarrollo sostenible han cobrado una importancia significativa en respuesta a la crisis climática global y a la necesidad de promover un desarrollo económico que sea ambiental y socialmente responsable. Los esfuerzos del país para implementar estas políticas están alineados con los ODS y reflejan un compromiso con la transformación hacia prácticas de producción y consumo sostenibles \cite{dormido2021fundar, sostenible2021argentina}.

\subsection{Desarrollo Productivo Verde}

Según FUNDAR, un laboratorio de políticas públicas argentino, es crucial que el Estado, el mercado y la sociedad civil trabajen conjuntamente para enfrentar los retos ambientales. Este enfoque colaborativo es esencial para diseñar políticas que no solo apunten a reducir las emisiones de gases de efecto invernadero sino que también fomenten la independencia y el crecimiento económico sostenible de Argentina \cite{dormido2021fundar}.

\subsection{Estrategia Nacional de Consumo y Producción Sostenibles}

La "Estrategia Nacional de Consumo y Producción Sostenibles" representa un esfuerzo significativo por parte del gobierno argentino para integrar la sostenibilidad en todos los aspectos de la cadena de producción y consumo. La estrategia incluye medidas en educación, regulaciones políticas, promoción de tecnologías sostenibles, gestión de recursos y sostenibilidad en la cadena de valor \cite{sostenible2021argentina}.

\subsubsection{Componentes Clave de la Estrategia}

\begin{itemize}
    \item \textbf{Educación y Sensibilización:} Incrementar la conciencia sobre la importancia de prácticas sostenibles.
    \item \textbf{Políticas Regulatorias:} Implementar leyes que incentiven el uso de prácticas y tecnologías sostenibles en la industria.
    \item \textbf{Promoción de Tecnologías Sostenibles:} Apoyar el desarrollo y adopción de tecnologías que minimicen el impacto ambiental.
    \item \textbf{Gestión Sostenible de Recursos:} Fomentar el uso eficiente de recursos y la reducción de residuos.
    \item \textbf{Integración de la Sostenibilidad en la Cadena de Valor:} Aplicar criterios de sostenibilidad en compras públicas y privadas.
\end{itemize}

La estrategia ha tenido impactos multidimensionales, destacando mejoras en la competitividad económica, conservación de ecosistemas y empoderamiento social. Sin embargo, los desafíos incluyen la necesidad de mayor participación del sector privado y la efectiva implementación de políticas a nivel local y regional \cite{sostenible2021argentina}.

La evaluación del impacto real sobre las emisiones de GEI y la efectividad general de las políticas sostenibles en Argentina es un tema de debate. A pesar de los avances, la implementación efectiva de estas políticas enfrenta desafíos significativos, incluyendo la falta de recursos financieros, la resistencia de sectores arraigados y la necesidad de inversiones en tecnologías limpias. La colaboración entre el gobierno, la industria y la sociedad civil será fundamental para alcanzar los objetivos ambientales y de desarrollo sostenible del país \cite{dormido2021fundar, sostenible2021argentina}.

\section{Integración de tecnologías para la trazabilidad}

\section{Tecnología blockchain para la trazabilidad}

\section{Aplicaciones de blockchain en la cadena de suministro}

\section{Tecnología blockchain para la sustentabilidad}

\end{document}
