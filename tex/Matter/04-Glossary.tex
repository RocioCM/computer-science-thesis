% Place here all the terms that need definition within your document.

\newglossaryentry{blockchain}
{
    name=blockchain,
    description={\textit{Blockchain} es una estructura de datos distribuida que mantiene un registro inmutable de transacciones o eventos mediante técnicas criptográficas que protegen contra la manipulación. La información se organiza en transacciones que son validadas y agrupadas en bloques. Cada bloque, junto con un puntero al bloque anterior, forma una cadena de transacciones interconectadas. Una vez que una transacción se ha añadido a la cadena, no puede ser modificada ni eliminada porque requeriría cambiar todos los bloques posteriores en la cadena, lo cual es computacionalmente impracticable debido a la distribución y la seguridad criptográfica de la red \textit{blockchain} \cite{rennock2018blockchain}}
}

\newglossaryentry{clienteservidor}
{
    name=cliente-servidor,
    description={El modelo \textit{Cliente-Servidor} es una arquitectura de red en la que las tareas se distribuyen entre proveedores de recursos o servicios, llamados servidores, y solicitantes de servicios, llamados clientes. Los clientes inician solicitudes de servicios y los servidores responden a estas solicitudes, gestionando recursos como bases de datos, archivos o aplicaciones. Este modelo permite la centralización de recursos y facilita la escalabilidad y mantenimiento del sistema. Este modelo es ampliamente utilizado en aplicaciones web y móviles, donde el cliente (navegador o aplicación) interactúa con el servidor para acceder a datos y funcionalidades}
}

\newglossaryentry{cleanarchitecture}
{
    name=Clean Architecture,
    description={\textit{Clean Architecture} es un patrón de diseño de software propuesto por Robert C. Martin que organiza el código en capas concéntricas, donde las dependencias apuntan hacia el centro. Este enfoque busca crear sistemas independientes de \textit{frameworks}, bases de datos, interfaces de usuario y elementos externos, facilitando el \textit{testing}, mantenimiento y evolución del software. Las capas internas contienen la lógica de negocio y son independientes de las capas externas que manejan aspectos técnicos como bases de datos o interfaces web}
}

\newglossaryentry{iot}
{
    name=IoT,
    description={\textit{Internet of Things} (IoT) es un paradigma tecnológico que permite la interconexión de dispositivos físicos cotidianos a través de Internet, dotándolos de capacidades de comunicación, monitoreo y control remoto. En el contexto de la trazabilidad de cadenas de suministro, los dispositivos IoT como sensores RFID, códigos QR y sensores ambientales recopilan y transmiten datos en tiempo real sobre la ubicación, estado y condiciones de los productos a lo largo de su ciclo de vida, facilitando la construcción de sistemas de trazabilidad integrales}
}

\newglossaryentry{http}
{
    name=HTTP,
    description={\textit{Hypertext Transfer Protocol} (HTTP) es el protocolo de comunicación fundamental de la Internet que define cómo se formatean y transmiten los mensajes entre clientes y servidores web. HTTP establece las reglas para el intercambio de recursos web como páginas HTML, imágenes, videos y datos de aplicaciones. En sistemas de trazabilidad \textit{blockchain}, HTTP es comúnmente utilizado para las comunicaciones entre interfaces web y APIs que interactúan con los contratos inteligentes y bases de datos}
}

\newglossaryentry{mvc}
{
    name=MVC,
    description={\textit{Model-View-Controller} (MVC) es un patrón de arquitectura de software que separa la lógica de una aplicación en tres componentes interconectados: el Modelo (datos y lógica de negocio), la Vista (interfaz de usuario) y el Controlador (intermediario que gestiona la entrada del usuario y coordina el modelo y la vista). Esta separación permite un desarrollo más organizado, facilita el mantenimiento del código y posibilita que diferentes equipos trabajen simultáneamente en los distintos componentes}
}

\newglossaryentry{hash}
{
    name=hash,
    description={Un \textit{hash} es una función criptográfica que transforma datos de entrada de cualquier tamaño en una cadena de caracteres de longitud fija, actuando como una ``huella digital'' única de los datos originales. En \textit{blockchain}, los \textit{hashes} son fundamentales para garantizar la integridad de los datos, ya que se utilizan para enlazar bloques secuencialmente. Los \textit{hashes} tienen la propiedad de que cualquier modificación en el contenido original resulta en un \textit{hash} completamente diferente, lo que hace que sea computacionalmente impracticable alterar datos sin detección}
}

\newglossaryentry{api}
{
    name=API,
    description={\textit{Application Programming Interface} (API) es un conjunto de protocolos, herramientas y definiciones que permiten que diferentes aplicaciones de software se comuniquen entre sí de manera estructurada. Las APIs definen métodos de solicitud, formatos de datos, convenciones y reglas que deben seguir las aplicaciones para intercambiar información. En sistemas de trazabilidad \textit{blockchain}, las APIs actúan como intermediarios que permiten a las aplicaciones web interactuar con los contratos inteligentes y consultar datos de la cadena de bloques}
}   

\newglossaryentry{uxui}
{
    name=UX/UI,
    description={UX/UI se refiere a \textit{User Experience/User Interface}, dos disciplinas complementarias en el diseño de productos digitales. La UI (Interfaz de Usuario) se enfoca en los elementos visuales e interactivos con los que los usuarios interactúan, como botones, menús y vistas. La UX (Experiencia de Usuario) abarca la experiencia completa del usuario al utilizar un producto, incluyendo la usabilidad, accesibilidad y satisfacción general}
}

\newglossaryentry{nodo}
{
    name=nodo,
    description={Un \textit{nodo} es un dispositivo conectado dentro de una red distribuida, como una \textit{blockchain}, que participa en la validación, almacenamiento y transmisión de datos. Los nodos pueden tener diferentes roles, como nodos completos que mantienen una copia completa de la \textit{blockchain} y validan todas las transacciones, o nodos ligeros que dependen de nodos completos para ciertas funciones. En una red \textit{blockchain}, los nodos trabajan conjuntamente para asegurar la integridad y seguridad del sistema mediante la verificación de transacciones y la participación en mecanismos de consenso}
}

\newglossaryentry{reddescentralizada}
{
    name=red descentralizada,
    description={Una red descentralizada es una arquitectura de comunicación donde no existe un punto central de control o falla, sino que el poder y la responsabilidad se distribuyen entre múltiples nodos participantes. A diferencia de los modelos centralizados donde un servidor central coordina las operaciones, en una red descentralizada cada nodo puede operar de forma independiente mientras mantiene sincronización con el resto de la red. \textit{Blockchain} es un ejemplo paradigmático de red descentralizada, donde múltiples participantes mantienen copias de la cadena de bloques y validan transacciones colectivamente}
}

\newglossaryentry{basededatos}
{
    name=base de datos,
    description={Una base de datos es un sistema organizado para almacenar, gestionar y recuperar información de manera estructurada. Permite a las aplicaciones almacenar datos persistentes, realizar consultas complejas y mantener la integridad de la información a través de reglas y restricciones. En sistemas de trazabilidad, las bases de datos permiten almacenar metadatos, información de sesiones de usuario y datos sobre el estado de los productos}
}

\newglossaryentry{apirest}
{
    name=API REST,
    plural=APIs REST,
    description={API REST (\textit{Representational State Transfer}) es un modelo arquitectónico para diseñar servicios web que utiliza los métodos estándar de HTTP (GET, POST, PUT, DELETE) para realizar operaciones sobre recursos identificados por URLs. Las APIs REST no mantienen estado entre solicitudes y utilizan una interfaz uniforme, lo que las hace escalables y fáciles de implementar. Para una aplicación descentralizada, las REST APIs proporcionan \textit{endpoints} que permiten a las aplicaciones \textit{frontend} consultar y actualizar información tanto de bases de datos tradicionales como de contratos inteligentes}
}

\newglossaryentry{endpoint}
{
    name=endpoint,
    description={Un \textit{endpoint} es una URL específica dentro de una API que representa un recurso o una funcionalidad particular a la que se puede acceder mediante solicitudes HTTP. Cada \textit{endpoint} define un punto de entrada para interactuar con el sistema, permitiendo operaciones como la obtención, creación, actualización o eliminación de datos. En un sistema de trazabilidad \textit{blockchain}, los \textit{endpoints} de una API REST pueden permitir a las aplicaciones \textit{frontend} consultar información de productos, registrar nuevas transacciones y comunicarse con contratos inteligentes en la \textit{blockchain}}
}


\newglossaryentry{frontend}
{
    name=frontend,
    description={El \textit{frontend} es la parte de una aplicación de software con la que los usuarios interactúan directamente, incluyendo la interfaz de usuario, elementos visuales, navegación y experiencia de usuario. Se ejecuta en el lado del cliente (navegador web o aplicación móvil) y se comunica con el \textit{backend} para obtener y enviar datos. En sistemas de trazabilidad \textit{blockchain}, el \textit{frontend} permite a los usuarios consultar información de productos, visualizar el historial de trazabilidad y realizar operaciones como registro de nuevos lotes o transferencias}
}

\newglossaryentry{backend}
{
    name=backend,
    description={El \textit{backend} es la parte de una aplicación de software que se ejecuta en el servidor y maneja la lógica de negocio, el procesamiento de datos, la autenticación, las conexiones a bases de datos y las comunicaciones con servicios externos. No es directamente visible para los usuarios finales, pero proporciona la funcionalidad que soporta el \textit{frontend}. En un sistema de trazabilidad \textit{blockchain}, el \textit{backend} gestiona las interacciones con los contratos inteligentes, procesa las transacciones \textit{blockchain} y administra las bases de datos complementarias}
}

\newglossaryentry{oop}
{
    name=POO,
    description={La Programación Orientada a Objetos (POO) es un paradigma de programación que organiza el software en torno a objetos que contienen tanto datos (atributos) como comportamiento (métodos). Los principios fundamentales de la POO incluyen encapsulamiento, herencia, polimorfismo y abstracción, que facilitan la reutilización de código, mantenimiento y escalabilidad}
}

\newglossaryentry{der}
{
    name=DER,
    description={Un Diagrama Entidad-Relación (DER) es una herramienta de modelado conceptual que representa gráficamente las entidades de un sistema de información, sus atributos y las relaciones entre ellas. Los DER son fundamentales en el diseño de bases de datos, ya que ayudan a visualizar la estructura lógica de los datos antes de su implementación real. En sistemas de trazabilidad, los DER modelan entidades como productos, lotes, ubicaciones y actores de la cadena de suministro, así como sus interrelaciones}
}

\newglossaryentry{json}
{
    name=JSON,
    description={\textit{JavaScript Object Notation} (JSON) es un formato ligero de intercambio de datos que es fácil de leer y escribir para humanos y simple de interpretar y generar para las computadoras. JSON se basa en la notación de objetos de JavaScript pero es independiente del lenguaje de programación. En APIs REST, JSON se utiliza para estructurar los datos enviados y recibidos entre clientes y servidores}
}

\newglossaryentry{qr}
{
    name=QR,
    description={\textit{Quick Response} (QR) es un tipo de código de barras bidimensional que puede almacenar información en una matriz de puntos y puede ser leído rápidamente por dispositivos móviles equipados con cámaras. Los códigos QR pueden contener varios tipos de datos como URLs, texto, números de teléfono o identificadores únicos. En sistemas de trazabilidad, los códigos QR se utilizan como identificadores físicos únicos que vinculan productos físicos con sus registros digitales, permitiendo a los usuarios acceder información del producto mediante el escaneo del código}
}

\newglossaryentry{css}
{
    name=CSS,
    description={\textit{Cascading Style Sheets} (CSS) es un lenguaje de definición de estilos para páginas web, utilizado para describir la presentación visual de documentos HTML y XML, incluyendo colores, fuentes, espaciado, disposición y animaciones. CSS permite separar el contenido de la presentación, facilitando el mantenimiento y la consistencia visual en aplicaciones web}
}

\newglossaryentry{url}
{
    name=URL,
    description={Una URL (\textit{Uniform Resource Locator}) es una dirección web que especifica la ubicación de un recurso en Internet y el protocolo utilizado para acceder a él. Las URLs incluyen componentes como el protocolo (HTTP/HTTPS), nombre del dominio, puerto, ruta (\textit{endpoint}) y parámetros de consulta. En aplicaciones web, las URL se utilizan para acceder al \textit{frontend} desde el cliente y para realizar solicitudes de recursos a APIs}
}

\newglossaryentry{uri}
{
    name=URI,
    description={Una \textit{Uniform Resource Identifier} (URI) es un identificador que define un nombre o ubicación de un recurso en Internet de manera única. Mientras que las URLs especifican tanto la ubicación como el método de acceso, las URIs pueden ser nombres o localizadores. Las URLs son un tipo específico de URL que indican cómo acceder al recurso. Por ejemplo, una URI puede representar un recurso abstracto sin una ubicación física}
}

\newglossaryentry{mecanismodeconsenso}
{
    name=mecanismo de consenso,
    description={Un mecanismo de consenso es un protocolo utilizado en redes distribuidas, como una \textit{blockchain}, que permite a los nodos de la red llegar a un acuerdo sobre el estado actual del sistema o sobre qué transacciones son válidas y deben ser agregadas al registro o cadena de bloques. El objetivo principal de un mecanismo de consenso es asegurar que todos los participantes de la red lleguen a un consenso o acuerdo sobre la verdad de los datos, incluso cuando algunos participantes puedan ser deshonestos o intenten manipular la red, sin depender de un servidor o nodo central. Un mecanismo de consenso eficaz debe ser seguro, resistente a la censura, tolerante a fallas y verificable en tiempo real \cite{diaz2022protocolos}}
}

\newglossaryentry{token}
{
    name=token,
    description={Un \textit{token}, en el contexto de \textit{blockchain}, es una unidad de valor que representa un activo digital y está asociado a una plataforma \textit{blockchain} en particular. Se suele utilizar como sinónimos de criptomoneda. Los tokens se pueden intercambiar, transferir, almacenar y utilizar en todas las aplicaciones construidas sobre una \textit{blockchain}. Los tokens pueden ser fungibles o no fungibles, dependiendo de si son intercambiables entre sí o son únicos}
}

\newglossaryentry{criptomoneda}
{
    name=criptomoneda,
    description={Una \textit{criptomoneda} es una moneda digital o virtual que utiliza criptografía para asegurar la integridad de las transacciones, controlar la emisión de nuevos fondos y verificar la transferencia de activos. Las criptomonedas operan de manera descentralizada en redes \textit{blockchain}, lo que las hace independientes de autoridades centrales como bancos o gobiernos. Ejemplos populares de criptomonedas incluyen Bitcoin y Ethereum}
}

\newglossaryentry{coverage}
{
    name=coverage,
    description={\textit{Coverage} o  cobertura de código es una métrica de testing que mide el porcentaje del código fuente que es ejecutado durante la ejecución de pruebas automatizadas. Una alta cobertura indica que la mayoría del código ha sido probado, lo que incrementa la confianza en la calidad del software y ayuda a identificar partes del código que no han sido testeadas. En el desarrollo de contratos inteligentes para sistemas de trazabilidad, el \textit{coverage} permite asegurar que todas las funciones críticas han sido probadas antes del despliegue en \textit{blockchain}, aunque no garantiza la ausencia de errores}
}

\newglossaryentry{software}
{
    name=software,
    description={El software es un conjunto de instrucciones, datos o programas utilizados para operar computadoras y ejecutar tareas específicas. Incluye aplicaciones, sistemas operativos, utilidades y herramientas de desarrollo que permiten a los usuarios interactuar con el hardware y realizar diversas funciones}
}


\newglossaryentry{ingenieriadesoftware}
{
    name=ingeniería de software,
    description={La ingeniería de software es una disciplina que aplica principios de ingeniería al al diseño, desarrollo, operación y mantenimiento de software. El objetivo es producir software de alta calidad que satisfaga las necesidades del usuario, sea mantenible y escalable. Esta disciplina se enfoca en gestionar la complejidad del desarrollo de sistemas de software a gran escala, integrando la programación con metodologías, herramientas y buenas prácticas}
}

\newglossaryentry{cadenadesuministro}
{
    name=cadena de suministro,
    description={La cadena de suministro constituye el entramado logístico, operativo y estratégico que permite el flujo de materiales, información y recursos desde la extracción de materias primas hasta la llegada de un producto al consumidor final. Involucra múltiples etapas y actores como proveedores, fabricantes, distribuidores, comerciantes y, en modelos circulares, gestores de residuos y reguladores. Su objetivo es garantizar que los bienes y servicios se produzcan y entreguen de manera eficiente, segura y rentable \cite{rodriguez2023modelamiento}}
}

\newglossaryentry{contratointeligente}
{
    name=contrato inteligente,
    plural=contratos inteligentes,
    description={Un contrato inteligente es un programa autoejecutado que implementa lógica de negocio directamente en una \textit{blockchain}, ejecutándose automáticamente cuando se cumplen condiciones predefinidas sin necesidad de intermediarios. Estos contratos son inmutables una vez desplegados y proporcionan transparencia y confianza en las transacciones ejecutadas \cite{buterin2013ethereum}}
}

\newglossaryentry{dapps}
{
    name=dApps,
    description={Una aplicación descentralizada (dApp) es una aplicación de software que se ejecuta en una red \textit{blockchain} en lugar de servidores centralizados. Las \textit{dApps} combinan contratos inteligentes en el \textit{backend} con interfaces de usuario tradicionales en el \textit{frontend}, permitiendo a los usuarios interactuar con servicios descentralizados de manera transparente}
}


\newglossaryentry{rfid}
{
    name=RFID,
    description={\textit{Radio Frequency Identification} (RFID) es una tecnología de identificación por radiofrecuencia que utiliza campos electromagnéticos para identificar y rastrear automáticamente etiquetas adheridas a objetos. Las etiquetas RFID contienen información almacenada electrónicamente que puede ser leída a corta distancia sin necesidad de contacto directo. En sistemas de trazabilidad de cadenas de suministro, RFID permite el seguimiento automatizado de productos a lo largo de su ciclo de vida, facilitando la implementación de sistemas de monitoreo integral en tiempo real}
}

\newglossaryentry{drs}
{
    name=DRS,
    description={\textit{Deposit Return Scheme} (DRS) o Sistema de Retorno de Depósito es un mecanismo económico donde los consumidores pagan un pequeño depósito al comprar productos con envases retornables, el cual es reembolsado cuando devuelven el envase vacío para su reciclaje o reutilización. Los DRS han demostrado ser altamente efectivos para aumentar las tasas de reciclaje de envases, especialmente botellas y latas de bebidas}
}

\newglossaryentry{trazabilidad}
{
    name=trazabilidad,
    description={La trazabilidad es la capacidad de identificar, registrar y seguir el historial, la aplicación o la ubicación de un producto, proceso o servicio a través de etapas identificadas de producción, procesamiento y distribución. En el contexto de economía circular, la trazabilidad permite monitorear el ciclo de vida completo de los productos desde su origen hasta su reutilización o reciclaje, proporcionando transparencia sobre el origen, procesamiento, transporte y destino final de los materiales, facilitando la implementación de prácticas sostenibles y la verificación de compromisos ambientales}
}

\newglossaryentry{economiacircular}
{
    name=economía circular,
    description={La economía circular es un modelo económico regenerativo que busca minimizar el desperdicio y maximizar el aprovechamiento de recursos mediante la reutilización, reparación, renovación y reciclaje de materiales y productos existentes. A diferencia del modelo lineal tradicional de ``extraer-producir-desechar'', la economía circular mantiene los productos y materiales en uso durante el mayor tiempo posible, extrayendo el máximo valor de ellos antes de su recuperación y regeneración. Este enfoque busca reducir la presión sobre los recursos naturales y minimizar el impacto ambiental de la producción y el consumo \cite{cerda2016economia}}
}

\newglossaryentry{sostenibilidad}
{
    name=sostenibilidad,
    description={La sostenibilidad es la capacidad de satisfacer las necesidades del presente sin comprometer la capacidad de las futuras generaciones para satisfacer sus propias necesidades. En el contexto empresarial e industrial, implica la implementación de prácticas que equilibren el crecimiento económico, la protección ambiental y el bienestar social}
}
